\documentclass[12pt,a4paper,openright,twoside]{report}
\usepackage[italian,english]{babel}
\usepackage{fancyhdr}
\usepackage{indentfirst}
\usepackage{newlfont}
\usepackage{pdfpages}
\usepackage{colortbl}
\usepackage{afterpage}
\usepackage{float}
\usepackage{graphicx}
\usepackage{listings}
\usepackage{subcaption}
\usepackage[font={small,it}]{caption}
\usepackage{emptypage}
\usepackage[hidelinks]{hyperref}
\usepackage{hhline}
\renewcommand{\lstlistingname}{Codice}
\definecolor{background}{HTML}{EEEEEE}
\lstdefinelanguage{nginx}{
	basicstyle=\footnotesize\ttfamily,
    numbers=left,
    numberstyle=\ttfamily,
    stepnumber=1,
    numbersep=8pt,
    showstringspaces=false,
    breaklines=true,
	belowskip=2em,
	aboveskip=2em,
    backgroundcolor=\color{background}
	}
	
\renewcommand*{\lstlistlistingname}{Elenco dei frammenti di codice}
\graphicspath{ {./images/} }
\oddsidemargin=30pt \evensidemargin=20pt
\pagestyle{fancy}\addtolength{\headwidth}{20pt}\setlength{\headheight}{15pt}
\renewcommand{\chaptermark}[1]{\markboth{\thechapter.\ #1}{}}
\renewcommand{\sectionmark}[1]{\markright{\thesection \ #1}{}}
\rhead[\fancyplain{}{\bfseries\leftmark}]{\fancyplain{}{\bfseries\thepage}}
\cfoot{}
\linespread{1.3}
\begin{document}
\includepdf[pages=-]{titlepage.pdf}
\clearpage{\pagestyle{empty}\cleardoublepage}
\begin{titlepage}
	\setcounter{page}{3}
	\thispagestyle{empty}
	\topmargin=6.5cm
	\raggedleft
	\large
	\em
	Alla migliore madre del mondo,\linebreak
	al miglior padre del mondo.\linebreak
	\newpage
	\clearpage{\pagestyle{empty}\cleardoublepage}
\end{titlepage}
\clearpage{\pagestyle{empty}\cleardoublepage}
\selectlanguage{italian}
\begin{abstract}
	\setcounter{page}{3}
	\pagenumbering{roman}
	Il Worldwide LHC Computing Grid (WLCG) è una collaborazione internazionale costituita da decine di 
	centri di calcolo distribuiti globalmente, la cui missione consiste nell'elaborazione 
	delle grandi quantità di dati prodotti dai maggiori esperimenti di Fisica delle Alte Energie, in particolare quelli al CERN di Ginevra. 
	
	Uno di questi centri di calcolo è ospitato presso il CNAF dell'Istituto Nazionale di Fisica Nucleare a Bologna, che contribuisce anche allo sviluppo 
	di middleware per la gestione dell'infrastruttura. 
	
	Molti componenti di tale middleware, che hanno funzionalità e scopi diversi, 
	richiedono un servizio di autorizzazione versatile e compatibile con i meccanismi 
	di autenticazione attualmente in uso, basati su token OAuth 2.0 e su certificati VOMS Proxy.
	
	In questa tesi si analizzerà l'architettura e l'implementazione di un proof-of-concept di un sistema di autorizzazione che soddisfi queste necessità, 
	enfatizzando l'integrazione delle tecniche di autenticazione citate. Per dimostrare la sua versatilità, verrà
	illustrato il processo di interfacciamento con un componente middleware attualmente in sviluppo al CNAF. 
	
	Il risultato finale ottenuto è un sistema che rispetta i vincoli richiesti e si integra facilmente con servizi eterogenei. 

\end{abstract}
\clearpage{\pagestyle{empty}\cleardoublepage}
\selectlanguage{italian}
\clearpage{\pagestyle{empty}\cleardoublepage}
\tableofcontents
\addtocontents{toc}{\protect\thispagestyle{empty}}
\listoffigures
\clearpage{\pagestyle{empty}\cleardoublepage}
\lstlistoflistings
\clearpage{\pagestyle{empty}\cleardoublepage}

\clearpage{\pagestyle{empty}\cleardoublepage}
\pagenumbering{arabic}

\chapter{Introduzione}
\lhead[\fancyplain{}{\bfseries\thepage}]{\fancyplain{}{\bfseries\rightmark}}
\textcolor{blue}{[cos'è la sicurezza? cos'è il controllo degli accessi?]}

\section{Autorizzazione e autenticazione}
\textcolor{blue}{[aggiungeer roba sulla storia?]}
I concetti di \textit{"autorizzazione"} e \textit{"autenticazione"} rappresentano due nuclei fondamentali nell'ambito 
della sicurezza informatica. Con \textit{autorizzazione} si intende la funzione che specifica i privilegi di accesso a determinate risorse o servizi. 
L'\textit{autenticazione}, invece, rappresenta l'atto di confermare la verità dell'identità di un utente.
Essenzialmente, il prcoesso di autorizzazione fa uso dei dati provenienti dal processo di autenticazione, e in base a questi stabilisce i 
permessi che un utente possiede.
\textcolor{blue}{[cos'è un IAM? come si collega con il progetto? chi è stato?]}
\textcolor{blue}{[com'è sviluppato attualmente un tipico sistema di autorizzazione?]}

\section{Contesto dell'operato}
\textcolor{blue}{[Migliorare coesione dei concetti]}
Open Policy Agent è un motore di policy open-source, che si occupa principalmente di definire e imporre le politiche di autorizzazione attraverso lo stack una applicazione.
Grazie a questo sistema, il processo di decision making nella fase di autorizzazione viene spostato dall'applicazione ad una compotente esterna. 
\newline
Le politiche vengono speicificate attraverso un linguaggio dichiarativo di alto livello chiamato \textit{Rego}.
\textcolor{blue}{[cos'è l'INFN?]}
\textcolor{blue}{[come mai questo progetto? tipi di autenticazione?]}
\textcolor{blue}{[obiettivo della tesi?]}

\section{Struttura del documento}
In questa tesi verrà esposta una leggera introduzione su alcuni concetti chiave sul contesto e sulle tecnologie in uso. 
Dopodiché, sarà mostrerà l'implementazione vera e propria dell'intera infrastruttura, aumentando il numero di feautures andando avanti
con i capitoli. Infine, verrà illustrata un'analisi delle performance e verranno fatte le ultime conclusioni.  


\chapter{Linee guida architetturali}
\lhead[\fancyplain{}{\bfseries\thepage}]{\fancyplain{}{\bfseries\rightmark}}
Durante la progettazione di un sistema di autorizzazione, è necessario considerare alcuni parametri fondamentali
 che ci permettono di valutarlo 
nel contesto in cui ci si trova.
\\ Quelli che hanno contribuito maggiormente sulle scelte progettuali sono stati:
\begin{itemize}
    \item la versatilità, cioé la capacità di interfacciarsi con sistemi di 
    funzionalità e struttura diversa.
    \item la manutenibilità, ovvero la facilità di apportare modifiche a sistema realizzato.
\end{itemize}
La prima è garantita attraverso la divisione in più componenti, cosicché modificare le regole d'accesso non richieda
un'alterazione del codice del complesso. 
\\ La seconda è ottenuta tramite la separabilità totale dal servizio che si intende proteggere,
 facilitandone la compatibilità con altre tecnologie.  
\\ Un'altra considerazione che ha influenzato l'architettura del sistema è stata la scelta del software per la gestione 
della fase decisionale dell'autorizzazione, che richiede una gestione delle due fasi dell'autorizzazione tramite l'uso di componenti seperati. 

\section{Componenti principali}

L'architettura dell'infrastruttura presenta uno scheletro formato da tre componenti di principali: 
\begin{itemize}
    \item il \textit{reverse proxy}
    \item il \textit{sistema delle politiche di accesso}
    \item il \textit{servizio applicativo}
\end{itemize}
Ciascuna di esse veste un ruolo importante nella gestione della richiesta, tuttavia solo le prime due influenzano il processo di autorizzazione. 

\subsection{Reverse proxy}
Il reverse proxy rappresenta l'unico entry-point dell'intero sistema di autorizzazione. Il compito di questo componente è di
mediare la comunicazione tra client e il servizio applicativo: ogni richiesta che viene fatta da un utente è obbligata ad attraversare 
il reverse proxy prima di giungere al server. 
\\ L'attraversamento del proxy è trasparente: al client apparirà di comunicare direttamente con il servizio, nonostante le richieste siano dirette al proxy e da esso propriamente gestite. 
\\ La mediazione fornita da questo elemento incorpora un aspetto decisionale, attraverso il quale viene effettivamente espresso il concetto di autorizzazione. 
Infatti, il reverse proxy inoltra o ignora le richieste al servizio applicativo, basandosi sull'output del sistema delle politiche di accesso.  
 
\subsection{Sistema delle politiche di accesso}
Il sistema delle politiche di accesso è la parte che determina come la richiesta dovrà essere gestita. 
\\ Per adempiere al suo ruolo, sfrutta le informazioni di identità dell'utente contenute nella richiesta, 
che devono essere propriamente formattate per essere comprese correttamente. 
\\ Dopodiché, applica le regole d'accesso e confronta i dati ricevuti con un database dei permessi, al quale ha accesso diretto.
Infine, restituisce in output una decisione, che stabilisce se l'accesso è consentito o meno.  

\subsection{Servizio applicativo} \label{serv_server}
Con servizio applicativo si intende il servizio che usufruisce del sistema di autorizzazione. 
Nonostante ciò, l'intero sistema gli è totalmente trasparente e non necessita di modifiche per interfacciarsi con esso.
\\Il suo ciclo d'esecuzione consiste nel gestire 
le richieste provenienti dagli utenti tramite
 il reverse proxy e formulare una risposta.
Questa viene inoltrata al mittente iniziale, senza che sia filtrata o analizzata, ma attraversando comunque il proxy.

\begin{figure}[h]
    \includegraphics[width=13cm]{topologia.png}
    \centering
    \caption{Rappresentazione concettuale del sistema}
    \label{topologia}
\end{figure}

\section{Topologia finale}
In base a quanto è stato esposto nei paragrafi precendenti, si ottiene una struttura topologica simile a quanto mostrato nella Figura \ref{topologia}. 
I numeri su ogni freccia dei flussi dati indicano la sequenza temporale delle trasmissioni di dati tra gli elementi. 
\\Ciascun componente è interpretabile come un processo separato in esecuzione sullo stesso server o su server diversi. 


\chapter{Implementazione}
\lhead[\fancyplain{}{\bfseries\thepage}]{\fancyplain{}{\bfseries\rightmark}}
La fase di implementazione delle basi del sistema rimane sicuramente la parte più corposa dello sviluppo del progetto.
\\ In questo capitolo verranno discussi in dettaglio gli strumenti e le tecniche
che hanno permesso la realizzazione dell'infrastruttura. Particolare attenzione sarà posta nella descrizione 
nella descrizione della parte operativa, attraverso l'illustrazione di frammenti di codice particolarmente significativi. 

\section{Software stack}
Con \textit{software stack} si intende l'insieme di sottosistemi software che permettono di realizzare una piattaforma completa.
Questi lavorano assieme per consentire la corretta esecuzione di un'applicazione oppure, come nel caso in questione, il 
funzionamento di un sistema.
Lo stack dell'infrastruttura è composto dai seguenti moduli, che saranno descritti nei paragrafi seguenti:
\begin{itemize}
    \item OpenPolicyAgent
    \item NGINX
    \item Docker
\end{itemize}
Ogni componente dell'infrastruttura sfrutta lo stack in modo diverso per svolgere il proprio compito. 


\subsection{OpenPolicyAgent}
\textit{OpenPolicyAgent} (abbreviato in OPA) è un software open source che permette di separare il processo di decision-making 
durante l'autorizzazione dall'applicazione rigorosa delle politiche.
\\ Consideriamo un servizio che richiede di stabilire se un utente possiede dei determinati diritti d'accesso.
OPA permette di gestire il problema decisionale analizzando le informazioni su di esso, ricevute attraverso un'interrogazione da parte del servizio. 
Il carico computazionale della risoluzione del problema viene dunque spostato ad un altro componente, che potrà essere interrogato da più 
agenti, senza la necessità di codice ridondante. 

\begin{figure}[h]
    \includegraphics[height=6cm]{opa-service.jpg}
    \centering
    \caption{Meccanismo di operazione di OpenPolicyAgent}
    \label{OPAWork}
\end{figure} 
La Figura \ref*{OPAWork} mostra in modo schematico la gestione tipica di una richiesta: un servizio invia un'interrogazione e OPA valuta i dati a sua disposizione per formulare una decisione e restituirla. 
Le informazioni della richiesta sono definiti in formato JSON, tuttavia sono strutturati in modo totalmente arbitrario. 
\\ Le politiche devono essere rappresentate attraverso il linguaggio dichiarativo 
\textit{Rego}, la cui sintassi si ispira al linguaggio 
di interrogazione \textit{Datalog}. 
Attraverso Rego, le regole si esprimono come delle asserzione sui dati mantenuti da OPA.
Il codice dedicato alle politiche diventa così molto conciso rispetto ad un loro equivalente in 
linguaggio imperativo. 
\\ OpenPolicyAgent è il software che alimenta \textcolor{blue}{[non so se il termine sia corretto]} interamente il sistema di politiche. Oltra a formulare una decisione,
 infatti, gestisce la ricezione delle informazioni e la trasmissione del risultatos.  


\subsection{NGINX}
\textit{NGINX} è un web server che incorpora numerose funzionalità, che facilitano l'impostazione del reverse proxy e del 
service server. 
È conosciuto per la sua stabilità e il suo consumo di risorse molto basso; contrariamente ai server tradizionali,  
non usa moltecipli thread per gestire le richieste, ma si limita ad un'architettura asincrona a eventi. Configurare il software per ottenere il comportamento desiderato diventa un processo relativamente semplice e intuitivo, e si
 limita alla creazione di un file con estensione ".config" con le clausole necessarie. 
 \\ NGINX permette di estendere le proprie funzionalità tramite la creazione di moduli esterni attraverso il linguaggio \textit{NJS}, 
 che consiste in una versione di JavaScript modificata. 

\subsection{Docker}
\textit{Docker} è una piattaforma che sfrutta la virtualizzazione a livello del sistema operativo per incorporare software 
in unità isolate chiamate \textit{container}, includendo anche le sue dipendenze. L'obiettivo di un container è quello di 
facilitare la distribuzione di un applicativo software, permettendone l'esecuzione a prescindere dal sistema operativo 
in uso dall'utente. 
\\ Docker fornisce lo strumento \textit{Compose}, che permette di definire delle applicazioni singole a più container, 
che comunicano fra loro su una rete privata. In questo modo è possibile creare facilmente un ambiente di sviluppo avente 
un funzionamento e struttura analoghe al sistema finale. Le caratteristiche dei container e i collegamenti fra questi vengono definite tramite un file di configurazione in YAML. 


\section{Implementazioni delle componenti}
L'intera struttura del sistema è definita attraverso il file di configurazione di Docker Compose. Ciò implica che ogni 
elemento è stato virtualizzato attraverso un container, che svolge il suo compito come un processo separato. 
\\ La topologia delle relazioni rispecchia quanto mostrato nella Figura \ref*{topologia}. 


\subsection{Sistema di politiche}
Il sistema di autorizzazione usa un approccio \textit{Role-Based Access Control} (abbreviato in RBAC), in cui il 
controllo degli accessi si basa sui diritti associati al ruolo dell'utente. 
Altre tecniche di controllo degli accessi possono essere implementate cambiando le politiche. Nonostante ciò, 
questo approccio risulta quello più consono alla dimostrazione delle capacità di OpenPolicyAgent.
\\ La descrizione dei ruoli e dei permessi associati ad essi è mantenuta in file in formato JSON, descritti come nel Codice \ref*{opa_roles}.
Ad ogni ruolo sono associate le operazioni che l'utente può svolgere.
\lstset{language=nginx}
\begin{lstlisting}[caption={Descrizione dei ruoli},captionpos=b,label=opa_roles]
    "roles" : {
        "/dev" : [
            "retrieve", 
            "submit"
        ],
        "/analyst" : [ "retrieve" ],
        "/admin" : [ 
            "retrieve", 
            "submit",
            "report",
            "image_request"
        ],
        "/moderator" : [ "report" ], 
        "/banned" : []
    }
\end{lstlisting}
Come già citato in precedenza, le politiche in OpenPolicyAgent sono definite attraverso il linguaggio Rego. L'unità base di una policy scritta in questo linguaggio è data 
dalla regola, che viene definita tramite una testa e un corpo contenuto in delle parentesi graffe. 
\begin{lstlisting}[caption={Policy in linguaggio Rego},captionpos=b,label=opa_policy]
    import input.http_method as http_method   
    default allow = false #abilita il default deny

    allow {
        check_permission
    }

    check_permission {
        all_permissions := data.roles[input.role][_]
        all_permissions == input.operation
    }
\end{lstlisting}
Il corpo di una regola contiene delle assegnazioni, operazioni logiche, oppure riferimenti ad altre regole. Una regola assume valore "true" 
solo se tutte le clausole interne sono eseguibili e risultano vere, altrimenti le viene assegnata il valore "undefined". Regole aventi stessa testa ma corpo diverso presentano come valore finale 
il risultato dato dall'OR logico fra i risultati di ciascuna di queste, con "undefined" che assume la stessa semantica di "false". 
Le regole possono essere annidate facendo sì che il significato di una regola dipenda dal valore di un'altre, in modo da formare una politica strutturata in un'organizzazione gerarchia. 
\\ La semplice politica mostrata nel Codice \ref*{opa_policy} è sufficiente ad implementare 
un controllo degli accessi basato su RBAC. 
La riga 2 permette di simulare un comportamento "default deny", facendo in modo che il valore di \texttt{allow} sia uguale a "false" 
nella situazione in cui le regole con tale testa siano "undefined".  \\ 
I confronti con collezioni di dati sono esprimibili in maniera concisa, come avviene nelle righe 9 e 10 del Codice \ref*{opa_policy}.
La variabile \texttt{all\_permission} contiene tutti i permessi del ruolo specificato nella query in input. 
Se l'operazione svolta dall'utente risulta in questa collezione, allora il valore di \texttt{check\_permission} sarà "true". 
\\ La risposta restituita alla componente che ha interrogato OPA consiste in un file in formato JSON contenente il valore 
delle regole alle radici dell'albero gerarchio, oppure di tutte le regole che hanno portato le radici ad avere un valore undefined. 


\subsection{Reverse proxy}
La scelta di inserire NGINX nel software stack è stata decisamente influenzata dagli strumenti che possiede per la realizzazione 
del reverse proxy. 
\\ Tra questi, NGINX fornisce la clausola \texttt{auth\_request} per fare appello a un sistema di decision-making in modo da valutare
i diritti d'accesso. 

\lstset{language=nginx}
\begin{lstlisting}[caption={Frammento di codice del reverse proxy},captionpos=b,label=nginx_rp]
    # operazione protetta
    location /operation {
        # appello al server di autorizzazione
        auth_request /authz; 
        proxy_pass http://service_server_web;
    }

    # verifica l'autorizzazione
    location /authz {
        internal;
        proxy_set_header Authorization $http_authorization;
        proxy_set_header Content-Length "";

        # passiamo gli header settati
        proxy_pass_request_headers on; 

        # sposta la gestione al modulo NJS  
        js_content auth_engine.authorize_operation;
    }

    # locazione usata per l'invio dei dati a OPA
    location /_opa {
        internal;
        proxy_pass_request_headers on; 
        proxy_pass http://opa:8181/;
    }
\end{lstlisting}
Una qualsiasi richiesta che accede ad un endpoint avente prefisso \texttt{/operation} induce il reverse proxy a controllare 
se l'utente possiede effettivamente l'autorizzazione ad accedere a tale servizio. Possiamo interpretare la locazione \texttt{/operation}
come quella associata alle risorse protette, che richiedono un controllo dell'accesso. 
\\ Attraverso il comando \texttt{auth\_request /authz},
il flusso d'esecuzione passa immediatamente al codice associato all'endpoint \texttt{/authz}. 
In questa fase, la richiesta HTTP viene condotta nel modulo NJS dedicato alla creazione della query e alla trasmissione di essa verso il sistema di politiche.
La riga 18 del Codice \ref*{nginx_rp} invoca questa funzionalità. \\
Il modulo NJS contiene il metodo \texttt{authorize\_operation}, che raccoglie le informazioni dagli header 
e le trascrive in formato JSON.
I dati sono quindi inseriti in un pacchetto HTTP, che verrà inviato all'endpoint \texttt{/\_opa}, 
 associato alla comunicazione con il sistema di politiche.
\lstset{language=nginx}
\begin{lstlisting}[caption={Frammento di codice del modulo NJS},captionpos=b,label=NJS]
    function authorize_operation(r) {

        // dati da inviare a OPA
        let opa_data = {
            "operation" : r.headersIn["X-Operation"],
            "role" : "/" + r.headersIn["X-Role"]
        }
    
        // pacchetto HTTP da inviare ad OPA
        var opts = {
            method: "POST",
            body: JSON.stringify(opa_data)
        };
        
        // gestisce la risposta di OPA
        r.subrequest("/_opa", opts, function(opa_res) {
    
            var body = JSON.parse(opa_res.responseText);
            
            // se body non c'e' o allow e' 
            // uguale a false, ritorna forbidden (403)
            if (!body || !body.allow) {
                r.return(403);
                return;
            }

            // altrimenti, ritorna il codice dato da OPA 
            // (che e' 200 in caso non vi siano errori)
            r.return(opa_res.status);
        });
    
    }
\end{lstlisting}
In base alla risposta del sistema di politche, il reverse proxy inoltrerà il pacchetto originale al service server con la keyword \texttt{proxy\_pass}, 
oppure terminerà la gestione della richiesta.
La parola chiave \texttt{internal} alle righe 10 e 23 del Codice \ref{nginx_rp}  garantisce che le locazioni 
associate all'interrogazione di OpenPolicyAgent possano essere accedute esclusivamente da sottorichieste interne al reverse proxy. 
\\ L'uso di un modulo esterno è necessario siccome la clausola \texttt{auth\_request} scarta il campo body della 
richiesta HTTP. Ciò implica che non è possibile inviare una richiesta contenente i dati della query nel suo corpo.
Dunque, è essenziale l'aggiunta di un passo aggiuntivo che converta gli header HTTP in dati in formato
JSON comprensibili da OPA. 


\subsection{Service server}
Come descritto nella Sezione \ref*{serv_server}, con service server si intende un servizio qualsiasi che sfrutta 
l'infrastruttura di autorizzazione. 
Siccome questa fase del progetto si focalizza sullo sviluppo corretto delle componenti dedicate 
al filtraggio delle richieste, il service server è dato da una API relativamente elementare. \\
L'API in questione ritorna una risposta generica con Status Code 200 se si accede correttamente all'endpoint \texttt{/operation}, 
mentre una richiesta a \texttt{/operation/image\_request} permette lo scaricamento di 
un'immagine. Siccome queste locazioni presentano il prefisso "operation", il loro accesso comporta l'uso 
del sistema di autorizzazione.  

La modalità web server di NGINX permette di implementare quanto appena descritto.
\lstset{language=nginx}
\begin{lstlisting}[caption={Frammento di codice del service server},captionpos=b,label=nginx_web]
    # invia 200 se si esegue l'operazione con successo
    location /operation {
        return 200;
    }

    # manda un'immagine al client
    location /operation/image_request {
        try_files $uri /test.jpg;
    }
\end{lstlisting}
La cluasola \texttt{return} invia al client uno specifico HTTP Status Code,  
Questa scelta ha facilitato lo sviluppo, permettendo di definire in poche righe di codice una API funzionante e adatta ai requisiti,  
come si può vedere nel Codice \ref*{nginx_web}.

\section{Cifratura SSL/TLS}
Spesso un sistema di autorizzaione richiede delle informazioni sensibili riguardo un utente per attuare il controllo 
degli accessi. 
Per proteggere il flusso dati da possibili intercettazioni è necessario che la comunicazione fra il servizio e l'utente venga cifrata. 
\\ Il protocollo più popolare che mette in sicurezza un collegamento \newline è \textit{SSL/TLS}, che si basa sulla cifratura a chiave asimmettrica. Il suo funzionamento necessita 
dei certificati X.509, e in questo modo garantiscono che la chiave pubblica di un host appartenga effettivamente ad esso. 
\\ Nel caso del progetto di questa tesi, solamente la connesione fra il reverse proxy e l'utente deve essere protetta, siccome nessun agente esterno
 all'infrastruttura possiede accesso diretto alle componenti.
NGINX supporta le connessioni SSL/TLS attraverso alcune configurazioni esplicite.

\begin{lstlisting}[caption={Configurazione di SSL},captionpos=b,label=nginx_ssl]
    server {
        listen 443 ssl;
        server_name servicecnaf.test.example;
        
        ssl_certificate     certs/star.test.example.cert.pem;
        ssl_certificate_key certs/star.test.example.key.pem;
        ...
\end{lstlisting}
L'impostazione del canale protetto attraverso NGINX avviene inserendo il parametro \texttt{ssl} nella definizione 
dei socket in ascolto del server. Il percorso del certificato pubblico e della chiave privata devono essere specificati 
rispettivamente nelle direttive \texttt{ssl\_certificate} e \texttt{ssl\_certificate\_key}. \\
Il Codice \ref*{nginx_ssl} mostra il setup della protezione SSL/TLS all'interno del reverse proxy del sistema di autorizzazione.
Allo stato attuale del progetto, i certificati sono dedicati esclusivamente al testing, quindi non sono validi. Va da sé che 
la sostituzione dei certificati stessi
con una controparte ufficialmente verificata è un processo banale.  


\section{Testing}
L'intera infrastruttura prende vita avviando i container di Docker Compose attraverso il comando \texttt{docker-compose up}.
\\ Per testare il sistema, è possibile inviare dei pacchetti HTTP al reverse proxy contenti nel proprio header le informazioni necessarie per formulare una query.
\\ \textit{cURL} è il software principale usanto in ambito di testing, siccome permette di inviare dati attraverso 
numerosi protocolli, tra cui anche HTTP. 
\\ Questo metodologia è stata applicata in continuità durante lo sviluppo dell'intero progetto. 

\chapter{Gestione dell'autenticazione}
\lhead[\fancyplain{}{\bfseries\thepage}]{\fancyplain{}{\bfseries\rightmark}}
Una delle motivazioni principale che ha portato alla creazione del progetto descritto in questa tesi 
sta nel creazione un sistema che supportasse 
un nuova tecnica di autenticazione e che fosse retrocompatibile con il metodo usato in precedenza dall'INFN.  
Queste due tecniche consistono rispettivamente nell'autenticazione basata su token e su certificati.    
\\ In questo capitolo si analizzeranno i due metodi diversi di autenticazione e si implementerà la gestione 
dei dati provenienti da loro.

\section{JSON Web Token}
\textcolor{blue}{[Cos'è un JWT? Com'è fatto?]}
\textcolor{blue}{[Quali sono le modifiche fatte al sistema? ]}
I \textit{JSON Web Token} (abbrev. in JWT) rappresentano uno standard per la definizione di un metodo 
compatto e sicuro per la trasmissione di dati in formato JSON.  
\\ Nella loro forma più diffusa, i JWT sono costituiti da tre parti:
\begin{itemize}
    \item L'\textit{header} contiene   
    \item Il \textit{payload}
    \item La \textit{signature}
\end{itemize}
Queste tre campi sono separati da dei punti, quindi un tipico JWT è nella forma:
\\ \centerline{\texttt{xxxxx.yyyyy.zzzzz}}
\\
OpenPolicyAgent supporta delle funzioni built-in per la decodifica e la verifica dei JWT.
con le regole usate per l'implementazione del RBAC nel Codice \ref*{opa_policy}. 

\section{Certificati VOMS Proxy}
Nell'ambito del calcolo distribuito, le organizzazioni virtuali (abbrev. VO) rappresentano una collezione di gruppi 
di invidui sparsi nel mondo che definite da una serie di regole riguardo la condivisione delle risorse.
\\ Il Virtual Organization Membership Service (abbrev. VOMS), è un servizio per il controllo degli accessi 
nei servizi distribuiti, che implementa
un database per ogni organizzazione virtuale (di fatto per ogni esperimento) all’interno del
quale sono mantenuti i dati relativi ad ogni singolo membro/utente


\chapter{Caso d'uso: interfacciamento con StoRM-Tape}
\lhead[\fancyplain{}{\bfseries\thepage}]{\fancyplain{}{\bfseries\rightmark}}
Durante le fasi di sviluppo iniziali, il servizio collegato al sistema di autenticazione era dato da una API
elementare. Quest'ultima non presentava delle funzionalità significative: la sua creazione era totalmente 
a sostegno del sistema di autorizzazione, dunque il servizio vero e proprio che la componente forniva non era 
di estrema rilevanza.  
\\ In questa sezione verrà mostrato il processo di interfacciamento del sistema con un servizio 
realistico e complesso, analizzando le modifiche necessarie per abilitare un controllo degli accesso
nei suoi endpoint. 
\section{Storm-Tape API}
\textit{STOrage Resource Manager} (o StoRM) è un servizio dedicato alla gestione della memoria 
di massa nei centri di computazione distribuita. \\ In relazione a questo, \textit{Storm-Tape API} è un progetto sviluppato
 attivamente all'INFN CNAF, che nella data di scrittura di questa tesi è in fase di Proof-of-Concept.
 \textcolor{blue}{[correggimi se sto sbagliando qui giaco]}
Consiste in un'implementazione della specifica {WLCG Tape REST API}, che offre un'interfaccia comune per permettere 
ai client di gestire la residenza dei dati mantenuti sui dischi nei GRID di tutto il mondo. 
\\ Il nome Storm-Tape deriva dal fatto che, nei centri di calcolo distribuiti,
 le informazioni vengono mantenute in dischi basati su nastro, in quanto permettono di memorizzare 
grandi moli di dati con un consto relativamente basso. Dunque, come si può intuire, questo servizio è dedicato esclusivamente 
ai sistemi che fanno uso di memoria basata su nastro. \\ 
L'API fornisce le seguenti funzionalità:


\section{Interfacciare il sistema con il servizio}


\textcolor{blue}{[Perché questo?]}
\textcolor{blue}{[Cos'è e come funziona storm-tape API?]}
\textcolor{blue}{[Quali sono le modifiche necessarie per far funzionare il sistema?]}
\textcolor{blue}{[Che conclusioni hai tratto?]}
\textcolor{blue}{[È facile cambiare? Come avviene in generale il cambiamento? Cosa se ne trae?]}



\chapter{Conclusioni}
\lhead[\fancyplain{}{\bfseries\thepage}]{\fancyplain{}{\bfseries\rightmark}}
Nei capitoli precedenti è stato illustrato l'intero processo di realizzazione del sistema, partendo dalla
progettazione architetturale e evolvendolo gradualmente tramite l'aggiunta di nuove funzionalità, che includono il supporto alla comunicazione crittografata 
e a differenti tecnologie di autenticazione. 
\\ Il Capitolo 6 ha mostrato la facilità 
con cui questo sistema riesce ad integrarsi con un servizio già esistente, senza richiedere una 
riprogrammazione totale dei componenti, ma modificando il modulo dedicato alla formattazione dei dati. 
\\ Nel presente capitolo verranno esposte alcune proposte di sviluppi futuri emerse durante l'implementazione, assieme a una riflessione
 sui risultati a progetto realizzato. 

\section{Sviluppi futuri}
Il progetto esposto in questa tesi fornisce una buona base per la creazione di un sistema di autorizzazione 
in cui il carico del processo di decision-making è affidato ad un componente esterno.  
\\ Nonostante ciò, sono molte le possibilità per cui quanto è stato creato può essere esteso e migliorato. 

\subsection{Modifica delle policy dall'esterno}
OpenPolicyAgent è uno strumento molto affascinante e con questo progetto abbiamo 
appena scalfito la superficie delle possibilità che offre. \\ Uno degli aspetti particolarmente interessanti 
sarebbe lo sviluppo di un'estensione del sistema per consentire la modifica delle policy connettendosi all'infrastruttura. 
Come è stato illustrato nel Capitolo 3, infatti, OPA fornisce un'API che espone degli 
endpoint CRUD, in modo da permettere la modifica, l'inserimento e l'eliminazione delle regole "on the fly".
\\Un'estensione del genere abiliterebbe un qualsiasi utente privilegiato a modificare le regole con estrema
 convenienza.   
\\ Seguendo questo pricipio, una possibile implementazione potrebbe impiegare l'uso di policy di OPA per applicare un controllo degli accessi 
riguardo la modifica delle policy stesse, il che rappresenterebbe un caso di studio decisamente curioso. 

\subsection{Controllo degli accessi basato su capability}
In un approccio al controllo degli accessi \textit{capability-based}, un utente possiede una lista di oggetti o risorse con una descrizione dei 
permessi associati a ciascuno di essi, chiamata capability. Quando un client vuole accedere ad una determinato servizio, mostra ad esso la rispettiva
 capability, e il servizio consentirà al client di eseguire le operazioni che rispettano i permessi elencati in questa.  
 Un esempio classico è l'associazione metaforica fra una capability e una chiave che apre una porta, dove la porta è la risorsa alla quale si vuole accede.  
\\ Più volte in questa tesi si è fatto riferimento al fatto che il sistema, per come è stato impostato, implementa una tecnica 
di controllo degli accessi RBAC, ossia basata sui ruoli.
\\ Nonostante ciò, le politiche possono essere riscritte per ottenere un sistema di autorizzazione con un approccio capability-based.
\\ Questa estensione sarebbe inoltre agevolata dal fatto che VOMS supporta un sistema del genere, come è osservabile dal campo "Capability" prensente nelle FQAN.

\subsection{Accesso diretto a OpenPolicyAgent da parte del servizio applicativo}
Un possibile miglioramento alla topologia del sistema consiste nella riconfigurazione del servizio applicativo 
per permettere di accedere al sistema delle politiche quando è necessario un controllo dell'accesso da parte dell'applicazione. 
\\ In questo modo, l'eventuale processo di autorizzazione interno al servizio sarebbe anch'esso gestito da OpenPolicyAgent, 
che incapsulerebbe tutte le politiche e le regole d'accesso dell'infrastruttura.

\section{Osservazioni finali}
L'obiettivo fondamentale di un proof-of-concept consiste nel dimostrare la realizzabilità 
di un sistema, mi ritengo soddisfatto del risultato finale ottenuto. L'infrastruttura ricalca l'obiettivo 
posto da questa tesi, realizzando un controllo efficace degli accessi e interfacciandosi 
con poche modifiche ad altri servizi.  
\\Vorrei infine dedicare queste ultime righe per esporre alcuni momenti importanti della mia esperienza personale nella concretizzazione del progetto. 
\\ Ho avuto il piacere di avere un piccolo spazio “mio” in un centro di ricerca, e di ricevere consigli e supporto da persone eccellenti per know-how, che mi hanno fatto sentire “collega”.
\\ Grazie a questa tesi, sono entrato in contatto con il mondo della computazione distribuita, che prima mi era totalmente estraneo. 
È stato molto interessante osservare i meccanismi che coinvolgono l'esecuzione di batch job, e non dimenticherò mai 
la quantità di rumore prodotta dai pod.   
\\ Infine, questa tesi mi ha dato l'opportunità di mettere alla prova quanto avevo
imparato durante questi anni di corso e trasformarli in competenze “sul campo”:
 ritengo che questa esperienza sia stata fondamentale per il mio percorso da studente di Informatica.  

\begin{thebibliography}{99}
	\addcontentsline{toc}{chapter}{Bibliografia}

	\bibitem{nginx_bib}
	Nginx, Inc.,
	\textit{NGINX},
	\url{https://www.nginx.com/}.

    \bibitem{nginx_doc}
	Nginx, Inc.,
	\textit{NGINX Documentation},
	\url{https://nginx.org/en/docs/}.

    \bibitem{VOMS_bib}
	Andrea Ceccanti, Francesco Giacomini, Enrico Vianello,
	\textit{VOMS},
	\url{https://italiangrid.github.io/voms/}.

	\bibitem{JWT_bib}
	IETF,
	\textit{RFC 7519},
	\url{https://www.rfc-editor.org/rfc/rfc7519}.

    \bibitem{opa_bib}
	OpenPolicyAgent,
	\textit{OpenPolicyAgent},
	\url{https://www.openpolicyagent.org/}.

    \bibitem{opa_doc}
	OpenPolicyAgent,
	\textit{OpenPolicyAgent Documentation},
	\url{https://www.openpolicyagent.org/docs/latest/}.

    \bibitem{rego_doc}
	OpenPolicyAgent,
	\textit{Policy Language},
	\url{https://www.openpolicyagent.org/docs/latest/policy-language/}.

	\bibitem{jwt_doc}
	auth0, \textit{Introduction to JSON Web Tokens},
	\url{https://jwt.io/introduction}. 

	\bibitem{njs_doc}
	NGINX,
	\textit{njs scripting language},
	\url{https://nginx.org/en/docs/njs/}.

	\textcolor{blue}{[da rivedere]}
    
    \bibitem{wlcg_doc}
	INFN CNAF,
	\textit{WLCg Tape API},
	\url{https://www.openpolicyagent.org/docs/latest/policy-language/}

\end{thebibliography}

\chapter*{Ringraziamenti}
Ringrazio il professor Ozalp Babaoglu per la sua disponibilità come relatore, nonostante fosse prossimo alla pensione. 
Sono molto grato di essere uno dei suoi ultimi tesisti della sua carriera. I suoi contributi nell'ambito dei sistemi operativi sono inestimabili.  


Ringrazio il professor Francesco Giacomini, per avermi dato l'accesso a una delle esperienze formative più importanti della mia vita. Lo ringrazio 
inoltre per avermi assistito attentamente nello sviluppo del progetto e nella scrittura di questa tesi. Grazie anche per la tua pazienza nei miei confronti.  

Ringrazio tutto il personale dell'INFN CNAF per aver reso l'esperienza ancora più gradevole e per avermi fatto visitare il centro di calcolo, trattandomi sempre come se fossi un loro collega.  

Un ringraziamento speciale va alla mia famiglia, per avermi sempre dato la spinta di andare avanti e per credere in me, senza negarmi mai nulla. 

Ringrazio tutti gli amici che ho conosciuto nel corso durante questi tre anni: Leon, Drif, Giaco, Baldo, Adriano, Donnoh, Vir, Matteo, Pino e tanti altri. Non avrei potuto chiedere dei migliori compagni di corso. 


Infine, ringrazio Leti, per essermi stata accanto nei momenti più bui e avermi dato tutto l'affetto che esiste in questo mondo, anche quando non me lo meritavo.  


\end{document}
