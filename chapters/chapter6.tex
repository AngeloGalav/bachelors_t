Nei capitoli precedenti è stato illustrato l'intero processo di realizzazione del sistema, partendo dalla
progettazione architetturale e evolvendolo gradualmente tramite l'aggiunta di nuove funzionalità, che includono il supporto alla comunicazione crittografata 
e a differenti tecnologie di autenticazione. 
\\ Il Capitolo 5 ha mostrato la facilità 
con cui questo sistema riesce ad integrarsi con un servizio già esistente, senza richiedere una 
riprogrammazione totale dei componenti, ma modificando il modulo dedicato alla formattazione dei dati. 
\\ Nel presente capitolo verranno esposte alcune proposte di sviluppi futuri emerse durante l'implementazione, assieme a una riflessione
 sui risultati a progetto realizzato. 

\section{Sviluppi futuri}
Il progetto esposto in questa tesi fornisce una buona base per la creazione di un sistema di autorizzazione 
in cui il carico del processo di decision-making è affidato ad un componente esterno.  
\\ Nonostante ciò, sono molte le possibilità per cui quanto è stato creato può essere esteso e migliorato. 

\subsection{Modifica delle policy dall'esterno}
OpenPolicyAgent è uno strumento molto affascinante e con questo progetto abbiamo 
appena scalfito la superficie delle possibilità che offre. \\ Uno degli aspetti particolarmente interessanti 
sarebbe lo sviluppo di un'estensione del sistema per consentire la modifica delle policy connettendosi all'infrastruttura. 
Come è stato illustrato nel Capitolo 3, infatti, OPA fornisce un'API che espone degli 
endpoint CRUD, in modo da permettere la modifica, l'inserimento e l'eliminazione delle regole "on the fly".
\\Un'estensione del genere abiliterebbe un qualsiasi utente privilegiato a modificare le regole con estrema
 convenienza.   
\\ Seguendo questo pricipio, una possibile implementazione potrebbe impiegare l'uso di policy di OPA per applicare un controllo degli accessi 
riguardo la modifica delle policy stesse, il che rappresenterebbe un caso di studio decisamente curioso. 

\subsection{Controllo degli accessi basato su capability}
In un approccio al controllo degli accessi \textit{capability-based}, un utente possiede una lista di oggetti o risorse con una descrizione dei 
permessi associati a ciascuno di essi, chiamata capability. Quando un client vuole accedere ad una determinato servizio mostra ad esso la rispettiva
 capability, e il servizio consentirà al client di eseguire le operazioni che rispettano i permessi elencati in questa.  
 Un esempio classico è l'associazione metaforica fra una capability e una chiave che apre una porta, dove la porta è la risorsa alla quale si vuole accede.  
\\ Più volte in questa tesi si è fatto riferimento al fatto che il sistema, per come è stato impostato, implementa una tecnica 
di controllo degli accessi RBAC, ossia basata sui ruoli.
\\ Nonostante ciò, le politiche possono essere riscritte per ottenere un sistema di autorizzazione con un approccio capability-based.
\\ Questa estensione sarebbe inoltre agevolata dal fatto che i JWT forniti da WLCG supportano un sistema del genere. Il claim \texttt{scope} \cite{wlcg_jwt}, infatti,
 è riservato al mantenimento delle capability dell'utente a cui il token è assciato. 

\subsection{Accesso diretto a OpenPolicyAgent da parte del servizio applicativo}
Un possibile miglioramento alla topologia del sistema consiste nella riconfigurazione del servizio applicativo 
per permettere di accedere al sistema delle politiche quando è necessario un controllo dell'accesso da parte dell'applicazione. 
\\ In questo modo, l'eventuale processo di autorizzazione interno al servizio sarebbe anch'esso gestito da OpenPolicyAgent, 
che incapsulerebbe tutte le politiche e le regole d'accesso dell'infrastruttura.

\section{Osservazioni finali}
L'obiettivo fondamentale di un proof-of-concept consiste nel dimostrare la realizzabilità 
di un sistema, mi ritengo soddisfatto del risultato finale ottenuto. L'infrastruttura ricalca l'obiettivo 
posto da questa tesi, realizzando un controllo efficace degli accessi e interfacciandosi 
con poche modifiche ad altri servizi.  
\\Vorrei infine dedicare queste ultime righe per esporre alcuni momenti importanti della mia esperienza personale nella concretizzazione del progetto. 
\\ Ho avuto il piacere di avere un piccolo spazio “mio” in un centro di ricerca, e di ricevere consigli e supporto da persone eccellenti per know-how, che mi hanno fatto sentire “collega”.
\\ Grazie a questa tesi, sono entrato in contatto con il mondo della computazione distribuita, che prima mi era totalmente estraneo. 
È stato molto interessante osservare i meccanismi che coinvolgono l'esecuzione di batch job: non dimenticherò mai 
la quantità di rumore prodotta dai pod.   
\\ Infine, questa tesi mi ha dato l'opportunità di mettere alla prova quanto avevo
imparato durante questi anni di corso e trasformarli in competenze “sul campo”:
 ritengo che questa esperienza sia stata fondamentale per il mio percorso da studente di Informatica.  