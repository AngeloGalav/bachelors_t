Nei capitoli precedenti è stato illustrato l'intero processo di realizzazione del sistema, partendo dalla
progettazione architetturale e evolvendolo gradualmente tramite l'aggiunta di nuove feature, che includono il supporto a 
differenti tecnologie di autenticazione e alla comunicazione crittografata. 
\\ Il Capitolo 6 ha mostrato la facilità 
con cui questo sistema riesce ad integrarsi con un servizio già esistente, senza richiedere una 
riprogrammazione totale delle componenti, ma modificando le policy esistenti e il modulo dedicato alla formattazione dei dati. 
\\ Questo capitolo è dedicato alle considerazioni suscitate dall'osservazione del progetto realizzato. 

\section{Sviluppi futuri}
Il progetto esposto in questa tesi fornisce una buona base per la creazione di un sistema di autorizzazione 
in cui il carico del processo di decision-making è affidato a una componente esterna.  
\\ Nonostante ciò, sono molte le possibilità per cui quanto è stato creato può essere esteso e migliorato. 

\subsection{Modifica delle policy dall'esterno}
OpenPolicyAgent è uno strumento molto affascinante e con questo progetto abbiamo 
appena scalfito la superficie delle possibilità che offre. Uno degli aspetti particolarmente interessanti 
sarebbe lo sviluppo di un'estensione del sistema per consentire la modifica delle policy connettendosi da una 
rete esterna all'infrastruttura. Come è stato illustrato nel Capitolo 3, infatti, OPA fornisce un'API che espone degli 
endpoint CRUD, in modo da permettere la modifica, l'inserimento e l'eliminazione delle regole "on the fly".
Un'estensione del genere abiliterebbe un qualsiasi utente privilegiato a modificare le regole con estrema
 convenienza.   
\\ Seguendo questo pricipio, una possibile implementazone potrebbe impiegare l'uso di policy di OPA per applicare un controllo degli accessi 
riguardo la modifica delle policy stesse, il che rappresenterebbe un caso di studio decisamente curioso. 

\subsection{Migliorie nella gestione dei VOMS Proxy}
Nel Capitolo 4 si è trattato di come il sistema gestisce i certificati VOMS Proxy. 
Come citato, il modulo al momento disponibile per la validazione dei VOMS Proxy necessita dell'uso di OpenResty, 
ossia una versione modificata da terze 
parti che include il supporto a estensioni programmate nel linguaggio Lua. Pertanto, non è supportata direttamente dagli sviluppatori di NGINX. 
\textcolor{blue}{[perché è negativo questo?]}
\\  Una possibile miglioramento dell'infrastruttura si potrebbe trovare nella reimplementazione dei moduli VOMS, 
traducendoli in programmi equivalenti scritti in NJS. Infatti, quest'ultimo è il linguaggio supportato ufficialmente per 
l'estensione di NGINX con funzionalità personalizzate.  
\\ In questo modo il reverse proxy 
risulterebbe totalmente affidato all'implementazione standard NGINX.

\section{Osservazioni finali}
Considerando che l'obiettivo di un Proof-of-Concept consiste nel dimostrare la realizzabilità 
di un sistema, mi ritengo soddisfatto del risultato finale ottenuto. L'infrastruttura ricalca l'obiettivo 
posto nell'introduzione della tesi, permettendo di svolgere un controllo degli accessi e interfacciandosi 
con poche modifiche ad altri servizi.  
\\ Ho deciso di dedicare queste ultime righe per parlare della mia esperienza personale nella concretizzazione del progetto. 
Grazie a questa tesi, sono stato esposto al mondo della computazione distribuita, che prima mi era totalmente estraneo. 
È stato molto interessante osservare i meccanismi che coinvolgono l'esecuzione di batch job, e non dimenticherò mai 
la quantità di rumore prodotta dai pod.   
\\ Più di tutto, questa tesi ha permesso di mettere alla prova le mie competenze. 
Non mi sarei mai accorto della mole di ricerca necessaria per la 
\\ Dunque, per concludere, ritengo che questa esperienza sia stata fondamentale per il mio percorso da studente di Informatica.  