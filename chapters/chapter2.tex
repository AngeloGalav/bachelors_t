L'organizzazione topologica dell'infrastruttura è probabilmente una delle caratteristiche più importanti dell'intero sistema. 
Il capitolo corrente sarà dedicato ad esporre le componenti principali, 
partendo inizialmente da una visione più generale. Gli ultimi paragrafi sono dedicati alla comunicazione tra ciascun componente, e all'implementazione di essi. 

\section{Reverse Proxy}
Il reverse proxy rappresenta il cuore pulsante dell'intero sistema di autorizzazione. Il compito di questo componente è di
 mediare la comunicazione tra client e server: ogni richiesta che viene fatta da un utente deve prima passare per il reverse proxy. 
 In questo modo, al client apparirà di comunicare direttamente con il server, nonostante le richieste siano dirette al proxy \textcolor{blue}{[<- cambiare questa frase]}. 
 Il reverse proxy   Si interfaccia con ciascun componente del sistema, e in ba  

\section{Componente decisionale}
\textcolor{blue}{[Migliorare coesione dei concetti]}
Open Policy Agent è un motore di policy open-source, che si occupa principalmente di definire e imporre le politiche di autorizzazione attraverso lo stack una applicazione.
Grazie a questo sistema, il processo di decision making nella fase di autorizzazione viene spostato dall'applicazione ad una compotente esterna. 
\newline
Le politiche vengono speicificate attraverso un linguaggio dichiarativo di alto livello chiamato \textit{Rego}.
Attraverso Rego, le richieste di autorizzazzione vengono 