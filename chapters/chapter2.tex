Durante la progettazione di un sistema di autorizzazione, è necessario considerare alcuni parametri fondamentali che ci permettono di valutarlo 
nel contesto in cui ci si trova.
\\ Quelli che hanno influito maggiormente sulle scelte progettuali sono stati:
\begin{itemize}
    \item la versatilità, cioé la capacità di interfacciarsi con sistemi di 
    funzionalità e struttura diversa.
    \item la manutenibilità, ovvero la facilità di apportare modifiche a sistema realizzato.
\end{itemize}
La prima è garantita attraverso la divisione in più componenti, così che modificare le regole d'accesso non richiede 
un'alterazione del codice del complesso. 
\\ La seconda è ottenuta tramite la separabilità totale dal servizio che si intende proteggere,
 facilitandone la compatibilità con altre tecnologie.   

\section{Componenti principali}

L'architettura dell'infrastruttura presenta uno scheletro formato da tre componenti di principali: 
\begin{itemize}
    \item il \textit{reverse proxy}
    \item il \textit{sistema di politiche}
    \item l' \textit{application server}
\end{itemize}
Ciascuna di esse veste un ruolo importante nella gestione della richiesta, tuttavia solo le prime due influenzano il processo di autorizzazione. 

\subsection{Reverse proxy}
Il reverse proxy rappresenta l'entry-point dell'intero sistema di autorizzazione. Il compito di questo componente è di
mediare la comunicazione tra client e application server: ogni richiesta che viene fatta da un utente è obbligata ad attraversare 
il reverse proxy prima di giungere al server. 
L'attraversamento del proxy è trasparente: al client apparirà di comunicare direttamente con il server, nonostante le richieste siano dirette al proxy e da esso propriamente gestite. 
\\ La mediazione fornita da questo elemento incorpora un aspetto decisionale, attraverso il quale viene effettivamente espresso il concetto di autorizzazione. 
Infatti, il reverse proxy inoltra o ignora le richieste all'application server, basandosi sull'output del sistema di politiche.  
 
\subsection{Sistema di politiche}
Il sistema di politiche è la parte che determina come la richiesta dovrà essere gestita. 
\\ Per adempiere al suo ruolo, sfrutta le informazioni di identità dell'utente contenute nella richiesta, 
che devono essere propriamente formattate per essere comprese correttamente. 
Dopodiché, applica le regole d'accesso e confronta i dati con i permessi, ai quali ha accesso diretto.
Infine, restituisce in output una decisione, che stabilisce se l'accesso è consentito.  

\subsection{Application server} \label{serv_server}
Con application server si intende il servizio che sfrutta il sistema di autorizzazione. Come tale,
 il suo ciclo d'esecuzione consiste nel gestire 
le richieste provenienti dagli utenti tramite
 il reverse proxy e formulare una risposta.
Questa viene inoltrata al mittente iniziale, senza che sia filtrata o analizzata, ma attraversando comunque il proxy.

\begin{figure}[h]
    \includegraphics[width=13cm]{topologia.png}
    \centering
    \caption{Rappresentazione concettuale del sistema}
    \label{topologia}
\end{figure}

\section{Topologia finale}
In base a quanto è stato esposto nei paragrafi precendenti, si ottiene una struttura topologica simile a quanto mostrato nella Figura \ref{topologia}. 
I numeri su ogni freccia dei flussi dati indicano la sequenza temporale delle trasmissioni di dati tra gli elementi. 
\\Ciascun componente è interpretabile come un processo separato in esecuzione sullo stesso server o su server diversi. 
