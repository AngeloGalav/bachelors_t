Durante la progettazione di un sistema di autorizzazione, è necessario considerare alcuni parametri fondamentali che ci permettono di valutarlo 
nel contesto in cui ci si trova.
Quelli che hanno influito maggiormente sulle scelte progettuali sono state la \textit{versatilità} (la capacità di interfacciarsi con sistemi di 
funzionalità e struttura diversa)
e la \textit{manutenibilità} (la facilità di apportare modifiche a sistema realizzato).
\\ Attraverso la divisione in più componenti, modificare le regole d'accesso non richiede un'alterazione del codice del
complesso. Inoltre, la separabilità totale dal servizio che si intende autorizzare permette di facilitare la compatibilità con altre tecnologie.   

\section{Componenti}

L'architettura dell'infrastruttura presenta uno scheletro formato da tre componenti di principali: il \textit{reverse proxy}, il \textit{sistema di politiche} e il \textit{service server}. 
Ciascuna di esse veste un ruolo importante nella gestione della richiesta, tuttavia solo le prime due influenzano il processo di autorizzazione. 

\subsection{Reverse Proxy}
Il reverse proxy rappresenta l'entry-point dell'intero sistema di autorizzazione. Il compito di questo componente è di
mediare la comunicazione tra client e service server: ogni richiesta che viene fatta da un utente dovrà prima attraversare il reverse proxy. 
In questo modo, al client apparirà di comunicare direttamente con il server, nonostante le richieste siano dirette al proxy e da esso propriamente gestite. 
\\ La mediazione fornita da questo elemento incorpora un aspetto decisionale, attraverso il quale viene espresso effettivamente il concetto di autorizzazione. 
Infatti, il reverse proxy inoltra o ignora le richieste al service server, basandosi sull'output di un sistema di politiche.  
 
\subsection{Sistema di politiche}
Il sistema di politiche è la parte che determina come la richiesta dovrà essere gestita. 
\\ Per adempiere al suo ruolo, sfrutta le informazioni di identità dell'utente contenute nella richiesta, che devono essere propriamente formattate per essere comprese correttamente. 
Dopodiché, le confronta con dati quali politiche e permessi, ai quali ha accesso diretto.
Infine, restituisce in output una decisione, che stabilisce se l'accesso è consentito.  

\subsection{Service server}
Con service server si intende il servizio che sfrutta il sistema di autorizzazione. Come tale, il suo ciclo d'esecuzione consiste nel gestire 
le richieste provenienti dagli utenti tramite
 il reverse proxy e formulare una risposta.
Questa viene indirezionata al mittente iniziale, senza che sia filtrata o analizzata, ma attraversando comunque il proxy.

\begin{figure}[h]
    \includegraphics[width=13cm]{topologia.png}
    \centering
    \caption{Rappresentazione concettuale del sistema}
    \label{topologia}
\end{figure}

\section{Topologia finale}
In base a quanto è stato esposto nei paragrafi precendenti, si ottiene una struttura topologica simile a quanto mostrato nella Figura \ref{topologia}. 
I numeri su ogni freccia indicano l'ordinamento temporale delle trasmissioni di dati tra ogni elemento. 
\\Ciascun componente è interpretabile come un processo separato in esecuzione sullo stesso server o su server diversi. 
