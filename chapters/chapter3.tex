bho.

\section{Software stack}
\subsection{OpenPolicyAgent}
\subsection{NGINX}
\subsection{Docker}
\section{Implementazioni delle componenti}
\subsection{Reverse proxy}
\subsection{Sistema di politiche}
\subsection{Service server}
Come descritto nel Capitolo 2, con service server si intende un servizio qualsiasi che sfrutta l'infrastruttura di autorizzazione. 
Siccome questa fase del progetto si focalizza sullo sviluppo corretto delle componenti dedicate 
al filtraggio delle richieste, il service server è dato da una API abbastanza elementare. \\
L'API in questione ritorna 200 se si accede correttamente all'endpoint \texttt{/operation}, 
mentre una richiesta a \texttt{/operation/image\_request} permette lo scaricamento di 
un'immagine. Siccome queste locazioni presentano il prefisso "operation", il loro accesso comporta l'uso 
del sistema di autorizzazione.

Per implementare quando appena descritto, si è utilizzato nuovamente il software NGINX, 
sfruttando la sua modalità webserver.
\lstset{language=nginx}
\begin{lstlisting}[caption={sda},captionpos=b,label=nginx_web]
    # invia 200 se si esegue l'operazione con successo
    location /operation {
        return 200;
    }

    # simula la richiesta di una risorsa al server
    location /operation/image_request {
        try_files $uri /test.jpg;
    }
\end{lstlisting}
Questa scelta ha facilitato lo sviluppo, permettendo di definire una API funzionante in poche righe di codice.  



\section{Testing}