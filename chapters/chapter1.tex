\textcolor{blue}{[cos'è la sicurezza? cos'è il controllo degli accessi?]}

\section{Autorizzazione e autenticazione}
\textcolor{blue}{[aggiungeer roba sulla storia?]}
I concetti di \textit{"autorizzazione"} e \textit{"autenticazione"} rappresentano due nuclei fondamentali nell'ambito 
della sicurezza informatica. Con \textit{autorizzazione} si intende la funzione che specifica i privilegi di accesso a determinate risorse o servizi. 
L'\textit{autenticazione}, invece, rappresenta l'atto di confermare la verità dell'identità di un utente.
Essenzialmente, il prcoesso di autorizzazione fa uso dei dati provenienti dal processo di autenticazione, e in base a questi stabilisce i 
permessi che un utente possiede.
\textcolor{blue}{[cos'è un IAM? come si collega con il progetto? chi è stato?]}
\textcolor{blue}{[com'è sviluppato attualmente un tipico sistema di autorizzazione?]}

\section{Contesto dell'operato}
\textcolor{blue}{[Migliorare coesione dei concetti]}
\textcolor{blue}{[cos'è l'INFN?]}
\textcolor{blue}{[come mai questo progetto? tipi di autenticazione?]}
\textcolor{blue}{[obiettivo della tesi?]}

\section{Struttura del documento}
In questa tesi verrà esposta una leggera introduzione su alcuni concetti chiave sul contesto e sulle tecnologie in uso. 
Dopodiché, sarà mostrerà l'implementazione vera e propria dell'intera infrastruttura, aumentando il numero di feautures andando avanti
con i capitoli. Infine, verrà illustrata un'analisi delle performance e verranno fatte le ultime conclusioni.  
