\textit{Autenticazione} e \textit{autorizzazione} rappresentano due concetti fondamentali nell'ambito della
 sicurezza informatica.
 \\ Con autenticazione si intende l'atto di confermare l'identità di un utente. 
 L'autorizzazione, invece, indica la funzione che specifica i privilegi di accesso a determinate risorse o servizi.
\\I due processi sono interamente interconnessi: l'autorizzazione fa uso dei dati provenienti 
dall'autenticazione e in base a questi stabilisce i 
permessi che un'entità possiede.
\\ In informatica, l'autorizzazione si basa sulle \textit{politiche di accesso} (o policy), che vengono utilizzate per determinare se 
una richiesta d'accesso a una risorsa da parte di un utente autenticato deve essere approvata o rifiutata.
\\ Il processo del controllo degli accessi è composto da due stadi principali: una fase decisionale, in cui si valutano i permessi dell'utente 
 formulando una decisione in base alle politiche di accesso del sistema,
e una fase esecutiva, in cui si applica effettivamente quanto è stato deciso. 

\section{Worldwide LHC Computing Grid}
Nell'ambito del calcolo distribuito, il \textit{grid computing} consiste 
nell'uso di una rete di risorse computazionali distribuite geograficamente che collaborano 
per raggiungere un determinato obiettivo.
\\ Il \textit{Worldwide LHC Computing Grid} (WLCG) \cite{wlcg_doc} è un progetto dato dalla
 collaborazione 
delle infrastrutture grid sparse in tutto il mondo, con l'obiettivo
di analizzare le grandi quantità di dati generati dal \textit{Large Hadron Collider} (LHC) \cite{cern_lhc}, 
l'acceleratore di particelle situato nel CERN di Ginevra. 
\begin{figure}[h]
    \includegraphics[width=10cm]{map.png}
    \centering
    \caption{Locazione geografica dei centri di calcolo del Worldwide LHC Computing Grid}
    \label{mappa}
\end{figure}

Ad oggi, il WLCG incorpora circa 170 centri di calcolo sparsi in 42 paesi. 
Uno di questi risiede nel centro CNAF dell'Istituto Nazionale di Fisica Nucleare (INFN) \cite{infn_cnaf} a Bologna,
 e rappresenta uno dei principali centri di calcolo di questa associazione. Come tale, il CNAF contribuisce alla
  alla gestione e allo sviluppo del middleware associato all'infrastruttura grid del WLCG. 


\section{Scopo del progetto}
Molti dei sistemi software sviluppati al CNAF richiedono un sistema di autorizzazione. Siccome si tratta di progetti aventi scopi e funzionalità diverse,
implementare 
il controllo degli accessi direttamente in essi risulterebbe ridondante e potrebbe portare a problemi di ambiguità 
se le politiche fossero implementate in modo diverso per ogni componente.
\\ Il progetto sviluppato in questa tesi consiste nell'esplorazione di una soluzione a questo problema, ovvero un proof-of-concept di 
un sistema di autorizzazione general-purpose e versatile. Siccome si interfaccerà potenzialmente con progetti in relazione al WLCG, 
il sistema deve essere compatibile
 con le tecniche di autenticazione adottate dai suoi servizi.
\\ Attualmente, il WLCG si trova in una fase di transizione dall'autenticazione basata su certificati proxy in favore all'uso dei token OAuth 2.0.
 Dunque, è necessario che il sistema sia compatibile con entrambi i metodi. 
\\Un altro vincolo riguarda l'implementazione: i componenti del sistema devono basarsi su strumenti software d'industria prontamente disponibili,
 in modo da assicurarne 
la stabilità e l'affidabilità.  

\section{Struttura del documento}
Questa tesi si focalizzerà su un'analisi dettagliata del sistema creato, aiutandosi con schemi e frammenti 
di codice a supporto della spiegazione.
\\ Il Capitolo 2 è dedicato alla descrizione architetturale, analizzandone i componenti. 
\\ Il Capitolo 3 mostra i dettagli implementativi del sistema motivandoli e i software sul quale si basa, con un accenno alle tecniche di testing. 
\\ Il Capitolo 4 spiega le tecniche di autenticazione supportate dall'infrastruttura e come è stata gestita la loro integrazione.  
\\ Il Capitolo 5 descrive il processo di interfacciamento con un servizio esistente. 
\\ Infine, il Capitolo 6 tratterà dei risultati di quanto è stato creato, proponendo 
dei possibili miglioramenti e aggiungendo una riflessione personale sul lavoro svolto. 