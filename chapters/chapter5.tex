Durante le fasi di sviluppo iniziali, il servizio collegato al sistema di autenticazione era dato da una API
elementare. Quest'ultima non presentava delle funzionalità significative: la sua creazione era totalmente 
a sostegno del sistema di autorizzazione, dunque il servizio vero e proprio che la componente forniva non era 
di estrema rilevanza.  
\\ In questa sezione verrà mostrato il processo di interfacciamento del sistema con un servizio 
realistico e complesso, analizzando le modifiche necessarie per abilitare un controllo degli accesso
nei suoi endpoint. 
\section{Storm-Tape API}
\textit{STOrage Resource Manager} (o StoRM) è un servizio dedicato alla gestione della memoria 
di massa nei centri di computazione distribuita. \\ In relazione a questo, \textit{Storm-Tape API} è un progetto sviluppato
 attivamente all'INFN CNAF, che nella data di scrittura di questa tesi è in fase di Proof-of-Concept.
 \textcolor{blue}{[correggimi se sto sbagliando qui giaco]}
Consiste in un'implementazione della specifica {WLCG Tape REST API}, che offre un'interfaccia comune per permettere 
ai client di gestire la residenza dei dati mantenuti sui dischi nei GRID di tutto il mondo. 
\\ Il nome Storm-Tape deriva dal fatto che, nei centri di calcolo distribuiti,
 le informazioni vengono mantenute in dischi basati su nastro, in quanto permettono di memorizzare 
grandi moli di dati con un consto relativamente basso. Dunque, come si può intuire, questo servizio è dedicato esclusivamente 
ai sistemi che fanno uso di memoria basata su nastro. \\ 
L'API fornisce le seguenti funzionalità:


\section{Interfacciare il sistema con il servizio}


\textcolor{blue}{[Perché questo?]}
\textcolor{blue}{[Cos'è e come funziona storm-tape API?]}
\textcolor{blue}{[Quali sono le modifiche necessarie per far funzionare il sistema?]}
\textcolor{blue}{[Che conclusioni hai tratto?]}
\textcolor{blue}{[È facile cambiare? Come avviene in generale il cambiamento? Cosa se ne trae?]}

