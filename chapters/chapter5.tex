Durante le fasi di sviluppo iniziali, il servizio collegato al sistema di autenticazione era dato da un'API
elementare e senza funzionalità significative.
\\ Tuttavia, per quanto un'impostazione simile possa essere utile ai fini dell'implementazione del sistema, non 
permette di valutare correttamente le caratteristiche di un proof-of-concept.
\\ In questa sezione verrà mostrato il processo di interfacciamento del sistema con un servizio 
realistico e complesso, analizzando le modifiche necessarie per abilitare il controllo degli accessi
nei suoi endpoint. 

\section{Storm-Tape}
Gli storage dei grid computazionali presentano generalmente un sistema di memoria basato su nastri (da cui deriva il nome Storm-Tape). 
Questi sistemi sono costituiti da una libreria di nastri etichettati con un codice a barre. Per accedere o modificare i dati di un file, 
un braccio robotico individua il relativo nastro e lo inserisce all'interno di un lettore.
\\ \textit{STOrage Resource Manager} (StoRM) è un servizio dedicato alla gestione della memoria 
di massa nei centri di computazione distribuita. \\ In relazione a questo, \textit{Storm-Tape} è un progetto sviluppato
 attivamente dalla sezione di Software Development dell'INFN CNAF, che nella data di scrittura di questa tesi è in fase di proof-of-concept.
Consiste in un'implementazione della specifica {WLCG Tape REST API}, che offre un'interfaccia comune per permettere 
ai client di gestire la residenza dei dati mantenuti negli storage dati dei grid. 
\\ Le funzionalità fornite dall'API sono illustrate nella Tabella \ref*{tab:table-name}. Ai comandi \texttt{STAGE} e \texttt{ARCHIVEINFO} è associato un endpoint,
 che si comporta in modo diverso in base al metodo HTTP della richiesta inviata ad esso. Ad esempio, la funzionalità \texttt{RELEASE}
  si ottiene inviando una richiesta con metodo\texttt{DELETE} all'endpoint \texttt{/stage}.

\begin{table}
\begin{center}
\begin{tabular}{ | m{7em} | m{18em} | } 
  \hline
  \textbf{Funzionalità} & \textbf{Descrizione} \\ 
  \hline
  \texttt{STAGE} & Richiede che i file mantenuti su un determinato nastro siano reso disponibili sul disco rigido. \\ 
  \hline
  \texttt{RELEASE} & Indica che i file precedentemente resi disponibili non sono più richiesti. \\ 
  \hline
  \texttt{ARCHIVEINFO} & Richiede informazioni riguardo il progresso della scrittura dei file su nastro. \\ 
  \hline
\end{tabular}
\caption{\label{tab:table-name} Insieme delle operazioni fornite dalla specifica WLCG Tape REST API}
\end{center}
\end{table}

\section{Interfacciare il sistema con il servizio}
L'interfacciamento del sistema con il servizio richiede delle alterazioni alla configurazione del reverse proxy. In particolare, 
il modulo NJS adibito all'invio delle informazioni ad OpenPolicyAgent deve analizzare l'endpoint e il metodo della richiesta per 
determinare il tipo di operazione che l'utente sta svolgendo.
\\ La prima modifica consiste nel cambiamento della location "protetta", che porta a un controllo dell'accesso, 
facendo sì che rifletti la locazione associata all'API di Storm-Tape.  
\\ Il metodo HTTP e l'URI della richiesta sono reperibili attraverso delle variabili built-in di NGINX e assegnabili 
a delle variabili generiche, in modo simile a quanto svolto
nel Codice \ref*{nginx_var} per l'estrazione delle informazioni dai VOMS Proxy. In questo modo, è possibile accedere ai due dati dal modulo NJS,
 e possono essere utilizzati per stabilire il carattere dell'operazione richiesta dall'utente. Il processo di creazione della query e il suo invio al
  sistema delle politiche rimane inalterato dal modulo originale. 
\\ Le regole delle politiche di OPA non richiedono modifiche, purché si voglia mantenere un controllo degli accessi basato su RBAC. Ciò implica che è possibile usare JWT e VOMS Proxy 
per autorizzare il client, senza l'aggiunta di ulteriori funzioni. 
\\ Nonostante ciò, il database contenente le informazioni sui ruoli deve essere riscritto in modo da essere coerente con 
le funzionalità che fornisce Storm-Tape. 

\subsection{Risultati}
Le modifiche richieste per proteggere il servizio tramite il sistema di autorizzazione sono poche e 
non influenzano in modo significativo le politiche. Questo fatto dimostra la versatilità del sistema, 
siccome richiede poche alterazioni per essere compatibile con un servizio di carattere molto diverso 
da quello iniziale.   

