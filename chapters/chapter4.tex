Una delle motivazioni principale che ha portato alla creazione del progetto descritto in questa tesi 
sta nella creazione un sistema che supportasse 
un nuova tecnica di autenticazione e che fosse retrocompatibile con il metodo usato in precedenza dall'INFN CNAF.  
Queste due tecniche consistono rispettivamente nell'autenticazione basata su token e su certificati.    
\\ In questo capitolo si analizzeranno i due metodi diversi di autenticazione e si implementerà la gestione 
dei dati provenienti da essi.

\section{JSON Web Token}
I \textit{JSON Web Token} (abbrev. in JWT) rappresentano uno standard per la definizione di un metodo 
compatto e sicuro per la trasmissione di dati in formato JSON. La sicurezza di questo metodo di scambio delle 
informazioni deriva dal fatto che i token possono essere firmati digitalmente, assicurando così integrità e confidenzialità dei dati.  
\\ Nella loro forma più diffusa, i JWT sono costituiti, in successione, da tre parti:
\begin{itemize}
    \item L'\textit{header}, che contiene le informazioni riguardo il tipo di token (ovvero JWT) e 
    l'algoritmo usato per la firma dei dati. 
    \item Il \textit{payload}, dove risiedono le informazioni che si vogliono scambiare. Tipicamente, una parte di questo campo è 
    è dedicata a delle dichiarazioni predefinite riguardo il client e la validità del token. 
    \item La \textit{signature}, che si ottiene firmando l'header e il payload con la chiave pubblica del destinatario.
\end{itemize}
Queste tre campi sono separati da dei punti, quindi un tipico JWT è definito nella forma:
\\ \centerline{\texttt{xxxxx.yyyyy.zzzzz}}
\\ L'uso più comune dei JSON Web Tokens consiste nel loro impiego come metodo di autenticazione. Generalmente, dopo che un utente si 
autentica a un servizio tramite le proprie credenziali, riceve un token. Questo assicura l'utente possiede dell'identità dell'utente, 
e definisce  \textcolor{blue}{[da cambiare]}. In questo modo, ogni operazione successiva all'autenticazione che richiede un controllo dell'accesso 
e segue l'autenticazione invierà una richiesta che contiene tale token, senza che sia necessario  
-
OpenPolicyAgent supporta delle funzioni built-in per la decodifica e la verifica dei JWT.
con le regole usate per l'implementazione del RBAC nel Codice \ref*{opa_policy}. 
\textcolor{blue}{[Cos'è un JWT? Com'è fatto?]}
\textcolor{blue}{[Quali sono le modifiche fatte al sistema? ]}

\section{Certificati VOMS Proxy}
Nell'ambito del calcolo distribuito, le organizzazioni virtuali (abbrev. VO) rappresentano una collezione di gruppi 
di invidui sparsi nel mondo che definite da una serie di regole riguardo la condivisione delle risorse.
\\ Il Virtual Organization Membership Service (abbrev. VOMS), è un servizio per il controllo degli accessi 
nei servizi distribuiti, che implementa
un database per ogni organizzazione virtuale (di fatto per ogni esperimento) all’interno del
quale sono mantenuti i dati relativi ad ogni singolo membro/utente
