Una delle motivazioni principale che ha portato alla creazione del progetto descritto in questa tesi 
sta nella creazione un sistema che supportasse 
l'autorizzazione tramite token e che fosse retrocompatibile con il metodo d'autenticazione usato in precedenza dall'INFN CNAF.  
Quest'ultimo consiste in una tecnica basata su certificati proxy, ovvero certificati che permettono di eseguire delle operazioni facendo le veci di un'altra entità. 
\\ In questo capitolo si analizzeranno i due metodi diversi di autenticazione e il processo d'implementazione della gestione 
dei dati provenienti da essi.

\section{JSON Web Token}
I \textit{JSON Web Token} (abbrev. in JWT) rappresentano uno standard per la definizione di un metodo 
compatto e sicuro per la trasmissione di dati in formato JSON. La sicurezza di questo metodo di scambio delle 
informazioni deriva dal fatto che i token possono essere firmati digitalmente, assicurando così integrità e confidenzialità dei dati.  
\\ Nella loro forma più diffusa, i JWT sono costituiti, in successione, da tre parti:
\begin{itemize}
    \item L'\textit{header}, che contiene le informazioni riguardo il tipo di token (ovvero JWT) e 
    l'algoritmo usato per la firma dei dati. 
    \item Il \textit{payload}, dove risiedono le informazioni che si vogliono scambiare. Tipicamente, una parte di questo campo è 
    è dedicata a delle dichiarazioni predefinite riguardo il client e la validità del token. 
    \item La \textit{signature}, che si ottiene firmando l'header e il payload con la chiave pubblica del destinatario.
\end{itemize}
Queste tre campi sono separati da dei punti, quindi un tipico JWT è definito nella forma:
\\ \centerline{\texttt{xxxxx.yyyyy.zzzzz}}
\\ L'uso più comune dei JSON Web Tokens consiste nel loro impiego come metodo di autenticazione.

\subsection{OAuth 2.0 e OpenID Connect}
\textit{OAuth 2.0} è un protocollo per l'autorizzazione nelle applicazioni software. Definisce uno standard per la gestione del controllo degli accessi
in un servizio HTTP, stabilendo come un client possa provare la propria identità attraverso l'uso di un token d'accesso, in sostituzione all'uso delle proprie credenziali. 
\\ \textit{OpenID Connect} (abbrev. OIDC) è un layer d'autenticazione costruito su OAuth 2.0, che permette ad un'applicazione di 
verificare l'identità di un utente e di ottenere delle semplici informazioni su di esso tramite un ID Token. Questo protocollo abilita il \textit{Single Sign-On} (abbrev. in SSO), 
che consiste nella possibilità di autenticare la propria identità su più servizi non necessariamente correlati senza dover immettere le proprie credenziali. 
\\OIDC sfrutta i JSON Web Token, che sono ottenibili in conformità con i metodi specificati da OAuth 2.0. Per questo motivo, spesso si usa il gergo
"token OAuth 2.0" per fare riferimento ai JWT usati nel contesto di OIDC.  
\\OIDC e OAuth 2.0 permettono di definire uno standard per l'accesso basato su token, che rimane un canone per molti servizi su Internet.

\subsection{Generazione dei tokens}
Durante la fase di sviluppo, i token sono stati generati con il software \textit{OIDC Agent}, che fornisce JWT conformi alla specifica di OpenID Connect.
Questo software usa un'interfaccia da linea di comando e crea dei token attraverso una configurazione che stabilisce l'issuer e alcune caratteristiche del token.
\\ Nel contesto del progetto, i token generati contengono un campo \texttt{wlcg.groups}, che rappresenta una lista dei gruppi a cui l'utente appartiene. 
La sigla WLCG sta per \textit{Worldwide LHC Computing Grid}, che è un'organizzazione data dalla collaborazione fra le infrastrutture di 
grid computazionali presenti in tutto il mondo. 


\subsection{Integrazione dei JWT}
L'integrazione dei JWT come metodo di autenticazione del sistema di autorizzazione viene decisamente semplifica 
dalla scelta di inserire OpenPolicyAgent nel software stack. 
Questo infatti fornisce delle funzioni built-in per la decodifica e la verifica dei JWT.
\\ Dunque, per abilitare ufficialmente il controllo degli accessi tramite i dati presenti nei token OAuth, 
è necessario alterare il modulo del reverse proxy dedicato allo scambio dei dati al sistema di politiche. In particolare, la modifica consiste 
nell'aggiunta di un campo contenente il token d'autorizzazione nella richiesta dedicata ad OpenPolicyAgent.
\\ Per fare in modo che le policy nel sistema di politiche possano effettivamente analizzare queste informazioni e formulare una decisione, 
si sfruttano le funzioni built-in menzionate all'inizio di questo paragrafo. 
\begin{lstlisting}[caption={Esempio di regole per la gestione dei token OAuth 2.0},captionpos=b,label=opa_jwt]
    # controllo dei permessi dell'utente
    check_permission_jwt {
        data.roles[payload["wlcg.groups"][_]][_]
                == input.operation
    }
    
    # validazione del tempo del token 
    check_time_validity {
        payload.exp <= time.now_ns()
    } else { # caso di token senza scadenza
        payload.iss == payload.exp
        payload.exp == payload.nbf
    }
    
    # formattazione e decodifica del token
    payload := p {
        [_, p, _] := io.jwt.decode(input.token)
    }
\end{lstlisting}
Come mostrato nel Codice \ref{opa_jwt}, \texttt{io.jwt.decode} è uno dei metodi che OPA offre per l'estrazione dei dati da un token.
Tramite quanto ottenuto, possiamo verificare se i gruppi associati all'utente possano effettivamente usufruire del servizio al quale sta tentando di accedere.
\\ Le nuove regole inserite verranno richiamate nella clausola \texttt{allow}, in modo da influenzare la decisione finale riguardo l'autorizzazione. 
\\ OpenPolicyAgent fornisce anche altre funzioni per la gestione dei JWT, che includono la verifica della firma del token, 
permettendo di progettare delle politiche complesse e che ricoprono molti casi d'uso. 

\section{Certificati VOMS Proxy}
Nell'ambito del calcolo distribuito, le \textit{organizzazioni virtuali} (abbrev. VO) rappresentano delle collezioni di gruppi 
di individui sparsi nel mondo, definiti da una serie di regole riguardo la condivisione delle risorse. 
\\Il Virtual Organization Membership Service (abbrev. VOMS) è un servizio per il controllo degli accessi 
nei servizi distribuiti, e implementa
un database per ogni organizzazione virtuale, all'interno del
quale sono mantenuti i dati relativi ad ogni singolo membro del complesso. Il servizio VOMS è utilizzato quotidiamente da 
migliaia di scienziati del mondo per autorizzare l'accesso alle risorse dei sistemi distribuiti, per usufruire sia dello storage che della 
loro potenza di calcolo. 
\\ In questo contesto, le informazioni e gli attributi di un utente che sono gestite da VOMS vengono inserite in un certificato \textit{VOMS Proxy}.
Dunque, quando un membro di una VO necessita di eseguire un'operazione che richiede autorizzazione (i.e. la sottomissione di un problema in relazione ad un esperimento), 
dovrà includere il certificato nella sua richiesta d'accesso. 
\\ I certificati VOMS Proxy, come indicato dal nome, vengono usati per autenticarsi attraverso le informazioni di un'altra entità, in questo caso una VO, ed eseguire le operazioni
per conto di questa. Possiede lo stesso formato di un certificato X.509, con l'eccezione dell'aggiunta di un campo chiamato \textit{attribute certificate} (abbrev. AC),
 contenente un altro certificato associato alla VO. 

\subsection{HTTPG}
Il sistema VOMS e i certificati VOMS Proxy sono impiegati principalmente per la comunicazione attraverso \textit{Grid Security Infrastructure} (abbrev. GSI), ovvero 
il sistema di autorizzazione utilizzato nei grid computazionali. 
\\Tutte le infrastrutture che usufruiscono dei certificati VOMS Proxy, e quindi anche i GSI, non permettono una trasmissione di dati attraverso HTTP; bensì, richiedono un protocollo chiamato \textit{HTTPG}.
Quest'ultimo si differenzia con i protocolli HTTP e HTTPS nell'handshake iniziale, dove oltre allo scambio delle normali informazioni di inizializzazione, il client invia un bit che determina la volontà di delegare l'operazione o meno alla VO.
Dopo questa fase, la comunicazione prosegue seguendo quanto specificato dal protocollo HTTP.    
\\ HTTPG non rappresenta uno standard per la comunicazione, e dunque di norma non è integrato nei servizi al di fuori di quelli dedicati ai grid.  

\subsection{Generazione dei certificati VOMS Proxy}
Il processo di generazione di un VOMS Proxy richiede l'applicazione \textit{VOMS Clients}, che contatta un'organizzazione virtuale per ricevere un attribute 
certificate, e lo inserisce in una copia della chiave pubblica dell'utente. Tutto ciò avviene attraverso l'uso del comando \texttt{voms-proxy-init}. 
\\Il servizio VOMS fornisce una VO associata al testing dei servizi, chiamata test.vo. I certificati usati durante lo sviluppo 
dell'integrazione a VOMS Proxy presentano un AC associato a questa VO, e presentano un tempo di validità di 24 ore.   

\subsection{Integrazione dei VOMS Proxy}
Per la creazione di servizi 
L'integrazione del supporto all'autorizzazione tramite certificati VOMS Proxy comporta la riconfigurazione del reverse proxy da un sottosistema.   
Siccome l'uso dei VOMS Proxy influenza il protocollo di comunicazione, la componente reverse proxy dovrà subire alcuni cambiamenti. 
\\ L'INFN CNAF ha sviluppato un modulo NGINX , è
\\Il modulo dedicato all'invio dei dati al sistema di politiche accede alle variabili create all'interno di OpenResty, 
e le inserisce nel pacchetto destinato a OpenPolicyAgent. 
\\ Per permettere che le informazioni contenute nei VOMS influenzino il processo di decision-making del controllo degli accessi,
 è necessaria l'aggiunta di regole aggiuntive. La loro semantica e sintassi sarà simile a quanto illustrato nel Codice \ref*{opa_policy},
tuttavia si farà riferimento ad una VO rispetto a un ruolo o a un gruppo.

