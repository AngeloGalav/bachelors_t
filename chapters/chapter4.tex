Uno tra i principali obiettivi che ha portato alla creazione del progetto descritto in questa tesi 
risiede nella creazione un sistema che supportasse 
l'autorizzazione tramite token e che fosse retro-compatibile con il metodo d'autenticazione usato in precedenza dai sistemi di autorizzazione dei grid computazionali.  
Quest'ultimo consiste in una tecnica basata su certificati proxy, ovvero certificati che permettono di eseguire delle operazioni facendo le veci di un'altra entità. 
\\ In questo capitolo si analizzeranno i due metodi diversi di autenticazione e il processo d'implementazione della gestione 
dei dati provenienti da essi.

\section{JSON Web Token}
I \textit{JSON Web Token} (abbrev. in JWT) rappresentano uno standard per la definizione di un metodo 
compatto e sicuro per la trasmissione di dati in formato JSON. La sicurezza di questo metodo di scambio delle 
informazioni deriva dal fatto che i token possono essere firmati digitalmente, assicurando così integrità e confidenzialità dei dati.  
\\ Nella loro forma più diffusa, i JWT sono costituiti, in successione, da tre parti:
\begin{itemize}
    \item L'\textit{header}, che contiene le informazioni riguardo il tipo di token (ovvero JWT) e 
    l'algoritmo usato per la firma dei dati. 
    \item Il \textit{payload}, dove risiedono le informazioni che si vogliono scambiare. Tipicamente, una parte di questo campo è 
    è dedicata a delle dichiarazioni predefinite riguardo il client e la validità del token. 
    \item La \textit{signature}, che si ottiene firmando l'header e il payload con la chiave pubblica del destinatario.
\end{itemize}
Queste tre campi sono separati da dei punti, quindi un tipico JWT si presetna nella seguente forma:
\\ \centerline{\texttt{xxxxx.yyyyy.zzzzz}}
\\ L'uso più comune dei JSON Web Tokens consiste nel loro impiego come metodo di autenticazione.

\subsection{OAuth 2.0 e OpenID Connect}
\textit{OAuth 2.0} è un protocollo per l'autorizzazione nelle applicazioni software. Definisce uno standard per la gestione del controllo degli accessi
in un qualsiasi servizio HTTP, stabilendo il modo in cui un client possa provare la propria identità attraverso l'uso di un token d'accesso, in sostituzione all'uso delle proprie credenziali. 
\\ \textit{OpenID Connect} (abbrev. OIDC) è un layer d'autenticazione costruito su OAuth 2.0, che permette ad un'applicazione di 
verificare l'identità di un utente e di ottenere delle semplici informazioni su di esso tramite un ID Token. Questo protocollo abilita il \textit{Single Sign-On} (abbrev. in SSO), 
che consiste nella possibilità di autenticare la propria identità su più servizi non necessariamente correlati senza dover immettere le proprie credenziali di volta in volta. 
\\OIDC sfrutta i JSON Web Token, che sono ottenibili in conformità con i metodi specificati da OAuth 2.0. Per questo motivo, spesso si usa il gergo
"token OAuth 2.0" per fare riferimento ai JWT usati nel contesto di OIDC.  
\\OIDC e OAuth 2.0 di fatto definiscono uno standard per l'accesso basato su token, che rimane un canone per molti servizi su Internet.

\subsection{Generazione dei tokens}
Durante la fase di sviluppo del sistema, i token sono stati generati con il software \textit{OIDC Agent}, che fornisce JWT conformi alla specifica di OpenID Connect.
Questo software usa un'interfaccia a linea di comando e crea dei token attraverso un'opportuna configurazione che stabilisce l'issuer e alcune caratteristiche del token.
\\ Nel contesto del progetto, i token generati contengono un campo \texttt{wlcg.groups}, che rappresenta una lista dei gruppi a cui l'utente appartiene. 
L'acronimo WLCG per esteso è \textit{Worldwide LHC Computing Grid}: si tratta di un'organizzazione nata dalla collaborazione fra le infrastrutture di 
grid computazionali presenti in tutto il mondo. 


\subsection{Integrazione dei JWT}
L'integrazione dei JWT come metodo di autenticazione del sistema di autorizzazione viene decisamente semplificata
dalla scelta di inserire OpenPolicyAgent nel software stack. 
Questo software infatti fornisce delle funzioni built-in per la decodifica e la verifica dei JWT.
\\ Un passaggio chiave per abilitare ufficialmente il controllo degli accessi tramite i dati presenti nei token OAuth, 
consiste nella modifica del modulo del reverse proxy dedicato allo scambio dei dati al sistema di politiche. In particolare, l'alterazione consiste 
nell'aggiunta di un campo contenente il token d'autorizzazione nella richiesta dedicata ad OpenPolicyAgent.
\\ Per fare in modo che le policy del sistema di politiche possano effettivamente analizzare queste informazioni e formulare una decisione, 
si sfruttano le funzioni built-in menzionate all'inizio di questo paragrafo. 
\begin{lstlisting}[caption={Esempio di regole per la gestione dei token OAuth 2.0},captionpos=b,label=opa_jwt]
    # controllo dei permessi dell'utente
    check_permission_jwt {
        data.roles[payload["wlcg.groups"][_]][_]
                == input.operation
    }
    
    # validazione del tempo del token 
    check_time_validity {
        payload.exp <= time.now_ns()
    } else { # caso di token senza scadenza
        payload.iss == payload.exp
        payload.exp == payload.nbf
    }
    
    # formattazione e decodifica del token
    payload := p {
        [_, p, _] := io.jwt.decode(input.token)
    }
\end{lstlisting}
Come mostrato nel Codice \ref{opa_jwt}, \texttt{io.jwt.decode} è uno dei metodi che OPA offre per l'estrazione dei dati da un token.
Grazie al risultato di questa operazione, possiamo verificare se i gruppi associati all'utente possano effettivamente usufruire del servizio al quale sta tentando di accedere.
\\ Le nuove regole inserite verranno richiamate nella clausola \texttt{allow}, in modo da influenzare la decisione finale riguardo l'autorizzazione. 
\\ OpenPolicyAgent fornisce anche altre funzioni per la gestione dei JWT, che includono la verifica della firma del token, 
permettendo di progettare delle politiche complesse in grado di soddisfare molti casi d'uso. 

\section{Certificati VOMS Proxy}
Nell'ambito del calcolo distribuito, le \textit{organizzazioni virtuali} (abbrev. VO) rappresentano delle collezioni di gruppi 
di individui sparsi nel mondo, definiti da una serie di regole riguardo la condivisione delle risorse. 
\\Il Virtual Organization Membership Service (abbrev. VOMS) è un servizio per il controllo degli accessi 
nei servizi distribuiti, che implementa
un database per ogni organizzazione virtuale, all'interno del
quale sono mantenuti i dati relativi ad ogni singolo membro del complesso. Il servizio VOMS è utilizzato quotidiamente da 
migliaia di scienziati nel mondo per autorizzare l'accesso alle risorse dei sistemi distribuiti, come ad esempio lo storage o la
loro potenza di calcolo. 
\\ In questo contesto, le informazioni e gli attributi di un utente che sono gestite da VOMS vengono inserite in un certificato \textit{VOMS Proxy}.
Pertanto, quando un membro di una VO necessita di eseguire un'operazione che richiede autorizzazione (i.e. la sottomissione di un problema in relazione ad un esperimento), 
dovrà includere il certificato nella sua richiesta d'accesso. 
\\ I certificati VOMS Proxy, come indicato dal nome stesso, vengono usati per autenticarsi attraverso le informazioni fornite da un'altra entità, in questo caso una VO, ed eseguire le operazioni
per conto di questa. I certificati possiedono lo stesso formato di un certificato X.509, con l'eccezione dell'aggiunta di un campo chiamato \textit{attribute certificate} (abbrev. AC),
 contenente un altro certificato associato alla VO. 
\\ Altre informazioni provenienti da questi certificati sono:
\begin{itemize}
    \item I \textit{Fully Qualified Attribute Names}, che sono delle stringhe di testo nella 
    forma \texttt{/VO[/group[/subgroup(s)]][/Role=role][/Capability=cap]} e indicano il ruolo di ogni utente nel gruppo e nei suoi sottogruppi. 
    \item Informazioni riguardo il certificato dell'utente e dell'ac, tra cui nazione e stato d'origine, organizzazione e common name.   
\end{itemize}
L'esteso set disponibile di informazioni che è possibile associare a ciascun utente permette di fornire
 una notevole flessibilità e specificità nella progettazione delle regole del controllo degli accessi. 

\subsection{HTTPG}
Il sistema VOMS e i certificati VOMS Proxy sono impiegati principalmente per la comunicazione attraverso \textit{Grid Security Infrastructure} (abbrev. GSI), ovvero 
il sistema di autorizzazione utilizzato nei grid computazionali. 
\\Tutte le infrastrutture che usufruiscono dei certificati VOMS Proxy, e quindi anche i GSI, non permettono una trasmissione di dati attraverso HTTP; bensì, richiedono un protocollo chiamato \textit{HTTPG}.
Quest'ultimo si differenzia con i protocolli HTTP e HTTPS nell'handshake iniziale, dove oltre allo scambio delle normali informazioni di inizializzazione, il client invia un bit che determina la volontà di delegare l'operazione o meno alla VO.
Dopo questa fase, la comunicazione prosegue seguendo quanto specificato dal protocollo HTTP.    
\\ HTTPG non rappresenta uno standard per la comunicazione, e dunque di norma non è integrato nei servizi al di fuori di quelli dedicati ai grid.  

\subsection{Generazione dei certificati VOMS Proxy}
Il processo di generazione di un VOMS Proxy richiede l'applicazione \textit{VOMS Clients}, che contatta un'organizzazione virtuale per ricevere un attribute 
certificate, e lo inserisce in una copia della chiave pubblica dell'utente. Tutto ciò avviene attraverso l'uso del comando \texttt{voms-proxy-init}. 
\\Il servizio VOMS fornisce una VO associata al testing dei servizi, chiamata test.vo. I certificati usati durante lo sviluppo 
dell'integrazione a VOMS Proxy presentano un AC associato a questa VO, e presentano un tempo di validità di 24 ore.   

\subsection{Integrazione dei VOMS Proxy}
L'integrazione del supporto all'autorizzazione tramite certificati VOMS Proxy comporta la riconfigurazione del reverse proxy, 
siccome l'uso di questi certificati richiede una modifica nel protocollo di comunicazione. 
\\ La sezione di Software Development dell'INFN CNAF ha sviluppato una patch di NGINX per abilitare la comunicazione 
tramite HTTPG ed estrarre automaticamente le informazioni dai VOMS Proxy.
Questo aggiornamento richiede un sottosistema basato su \textit{OpenResty}, un branch di NGINX che supporta moduli scritti in linguaggio Lua. 
Il processo di configurazione di OpenResty
 rimane pressocché lo stesso del software originale, dunque i file di configurazione e le impostazioni sono riutilizzabili 
senza che vi siano alterazioni. 
Si è deciso quindi di reimplementare il reverse proxy usando questa versione modificata di NGINX, sfruttando la patch per la gestione dei VOMS Proxy. 
\\ Lo sviluppo del supporto ai certificati VOMS consiste principalmente nell'inserimento da parte del reverse proxy 
delle informazioni estratte dal certificato in variabili interne ad OpenResty. 

\begin{lstlisting}[caption={Assegnazione di variabili in OpenResty},captionpos=b,label=openresty_var]
    location /operation/ {
        ...
        set $generic_var $voms_user;
        set $generic_var $voms_vo;
        ...
    }
\end{lstlisting}
In questo modo, il modulo NJS dedicato all'invio dei dati al sistema di politiche accede alle variabili create all'interno di OpenResty, 
e le inserisce nel pacchetto destinato a OpenPolicyAgent. 
\\Un esempio di assegnazione di valori in variabili interne a 
OpenResty è visibile nel Codice \ref{openresty_var}.
\\ Per permettere che le informazioni contenute nei VOMS influenzino il processo di decision-making del controllo degli accessi,
 è necessario l'inserimento nelle policy di regole aggiuntive. La loro semantica e sintassi sarà simile a quanto illustrato nel Codice \ref*{opa_policy},
tuttavia si farà riferimento ad una VO rispetto a un ruolo o a un gruppo.
\\ \\ Il sistema di autorizzazione, con l'aggiunta del supporto all'autorizzazione tramite token OAuth e certificati VOMS, riesce ad ottenere informazioni 
da due tecniche di autenticazione molto diverse e a formulare delle decisioni legate al controllo dell'accesso, senza che queste siano mutualmente esclusive. 

