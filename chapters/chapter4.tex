Una delle motivazioni principale che ha portato alla creazione del progetto descritto in questa tesi 
sta nel creazione un sistema che supportasse 
un nuova tecnica di autenticazione e che fosse retrocompatibile con il metodo usato in precedenza dall'INFN.  
Queste due tecniche consistono rispettivamente nell'autenticazione basata su token e su certificati.    
\\ In questo capitolo si analizzeranno i due metodi diversi di autenticazione e si implementerà la gestione 
dei dati provenienti da loro.

\section{JSON Web Token}
\textcolor{blue}{[Cos'è un JWT? Com'è fatto?]}
\textcolor{blue}{[Quali sono le modifiche fatte al sistema? ]}
I \textit{JSON Web Token} (abbrev. in JWT) rappresentano uno standard per la definizione di un metodo 
compatto e sicuro per la trasmissione di dati in formato JSON.  
\\ Nella loro forma più diffusa, i JWT sono costituiti da tre parti:
\begin{itemize}
    \item L'\textit{header} contiene   
    \item Il \textit{payload}
    \item La \textit{signature}
\end{itemize}
Queste tre campi sono separati da dei punti, quindi un tipico JWT è nella forma:
\\ \centerline{\texttt{xxxxx.yyyyy.zzzzz}}
\\
OpenPolicyAgent supporta delle funzioni built-in per la decodifica e la verifica dei JWT.
con le regole usate per l'implementazione del RBAC nel Codice \ref*{opa_policy}. 

\section{Certificati VOMS Proxy}
Nell'ambito del calcolo distribuito, le organizzazioni virtuali (abbrev. VO) rappresentano una collezione di gruppi 
di invidui sparsi nel mondo che definite da una serie di regole riguardo la condivisione delle risorse.
\\ Il Virtual Organization Membership Service (abbrev. VOMS), è un servizio per il controllo degli accessi 
nei servizi distribuiti, che implementa
un database per ogni organizzazione virtuale (di fatto per ogni esperimento) all’interno del
quale sono mantenuti i dati relativi ad ogni singolo membro/utente
