\documentclass{beamer}
\usepackage[utf8]{inputenc}
\setbeamertemplate{navigation symbols}{}

\useoutertheme{infolines}
\usetheme{Frankfurt}
\usecolortheme{default}
\definecolor{ao(english)}{rgb}{0.0, 0.5, 0.0}

\graphicspath{ {../images/} }

%------------------------------------------------------------
%This block of code defines the information to appear in the
%Title page
\title[] %optional
{Analisi e implementazione di un sistema di
autorizzazione compatibile con token
OAuth 2.0 e certificati VOMS}

\author[Angelo Galavotti] % (optional)
{Presentata da Angelo Galavotti\\ 
Relatore: Ozalp Babaoglu\\
Correlatore: Francesco Giacomini}


\institute[] % (optional)
{
  Corso di Laurea in Informatica\\
  Alma Mater Studiorum Università di Bologna
}

\date[12 Ottobre 2022] % (optional)
{12 Ottobre 2022}

%End of title page configuration block
%------------------------------------------------------------

\begin{document}

%The next statement creates the title page.
\frame{\titlepage}


%---------------------------------------------------------
%This block of code is for the table of contents after
%the title page
\begin{frame}
\frametitle{Sommario}
\tableofcontents
\end{frame}
%---------------------------------------------------------


\section{Introduzione}

%---------------------------------------------------------
%Changing visivility of the text
\begin{frame}
\frametitle{Background}

\begin{itemize}
  \item Il Worldwide LHC Computing Grid (\textbf{WLCG}) è una collaborazione mondiale di \textbf{centri di calcolo}
  finalizzata all'\textbf{analisi} e all'\textbf{elaborazione} dei dati generati dall'\textbf{acceleratore di particelle} del \textbf{CERN} di Ginevra.
  \\~\
  \item Uno di questi centri di calcolo è situato alla sede CNAF di Bologna dell'INFN, e si occupa
   anche dello \textbf{sviluppo del middleware} dell'infrastruttura \textbf{WLCG}.
   \\~\ 
  \item Molte componenti di tale middleware necessitano di un \textbf{sistema di autorizzazione}. 

\end{itemize}
\end{frame}

%---------------------------------------------------------


%---------------------------------------------------------
%Example of the \pause command
\begin{frame}
  \frametitle{Obiettivo}
  L'obiettivo del progetto consiste nella creazione di un PoC di un \textbf{sistema di autorizzazione} che sia \textbf{versatile} e \textbf{compatibile} 
  con i \textbf{metodi di autenticazione} attualmente adottati dal WLCG.
\end{frame}
%---------------------------------------------------------

\section{Implementazione del sistema}

%---------------------------------------------------------
%Highlighting text
\begin{frame}
\frametitle{Struttura}
L'infrastruttura presenta tre componenti principali, ciascuno in esecuzione all'interno di un container \textbf{Docker}:
\begin{itemize}
  \item \textbf{Sistema delle politiche di accesso}
  \item \textbf{Reverse proxy}
  \item \textbf{Servizio applicativo}
\end{itemize}

La separazione in componenti permette di \textbf{evitare ambiguità} durante il controllo degli accessi e di garantire la \textbf{manutenibilità}. 

\begin{figure}[h]
  \includegraphics[width=8cm]{topologia.png}
  \centering
\end{figure}
\end{frame}

\begin{frame}
  \frametitle{OpenPolicyAgent}
  \begin{itemize}
    \item \textbf{OpenPolicyAgent} è il sottosistema software alla base del \textbf{sistema delle politiche} di accesso.
    \item È un \textbf{policy engine} che permette l'\textbf{applicazione di politiche} e regole scritte in linguaggio Rego.
    \item Le regole implementate in questo progetto attuano un \textbf{modello RBAC} per il controllo degli accessi.
  \end{itemize}  
  
  \begin{figure}[h]
    \includegraphics[width=6cm]{opa-service.jpg}
    \centering
  \end{figure}
  
\end{frame}

\begin{frame}
  \frametitle{NGINX}
  \begin{itemize}
    \item \textbf{NGINX} è il software utilizzato per l'implementazione del \textbf{reverse proxy} e del \textbf{servizio applicativo}.
    \item È conosciuto per la sua \textbf{stabilità}, \textbf{efficienza} e \textbf{semplicità} di configurazione.
    \item Contiene direttive built-in che facilitano l'implementazione del reverse proxy e l'\textbf{interrogazione del sistema delle politiche} di accesso. 
    \item Le sue funzionalità possono essere estese tramite la scrittura di \textbf{procedure} in \textbf{linguaggio NJS}. 
  \end{itemize}  
  
  \begin{figure}[h]
    \includegraphics[width=4cm]{NGINX-logo.png}
    \centering
  \end{figure}
  
\end{frame}

\section[Autenticazione]{Gestione dei metodi di autenticazione}

\begin{frame}
  \frametitle{Gestione dei metodi di autenticazione}
  \begin{itemize}
    \item Un requisito chiave del sistema è la compatibilità con i \textbf{metodi di autenticazione} attualmente adottati dal \textbf{WLCG}.
    \\~\
    \item Le procedure di autenticazione del WLCG sono in \textbf{fase di transizione}:
    \begin{itemize}
      \item 
      attualmente si basano su \textbf{certificati VOMS Proxy} ed evolveranno verso l'uso di \textbf{JSON Web Token}. \\~\
    \end{itemize}
    
    \item È necessario che il PoC sia compatibile con entrambi i metodi. 
  \end{itemize}  
\end{frame}

\begin{frame}
  \frametitle{JSON Web Token}
  \begin{itemize}
    \item I JSON Web Token (\textbf{JWT}) definiscono un metodo \textbf{compatto} e \textbf{sicuro} per la trasmissione di dati
     in \textbf{formato JSON}.
    \item Sono alla base dallo standard \textbf{OpenID Connect}, basato sul protocollo di autorizzazione \textbf{OAuth 2.0}. 
    \item Si definiscono come una stringa di caratteri, in cui si riconoscono \textbf{tre parti separate da un punto}. 
  \end{itemize}  
  \hfill
  \centerline{\texttt{\textcolor{ao(english)}{[header]}.\textcolor{red}{[payload]}.\textcolor{blue}{[signature]}}}
  \begin{itemize}
    \item OpenPolicyAgent fornisce delle \textbf{funzioni built-in} per la decodifica e la verifica dei JWT. 
  \end{itemize}  
\end{frame}

\begin{frame}
  \frametitle{Certificati proxy VOMS}
  \begin{itemize}
    \item I certificati \textbf{VOMS Proxy} hanno lo stesso formato di un certificato \textbf{X.509}, ma aggiungono il campo \textbf{Attribute Certificate}.
    \item Questo campo contiene il certificato di una \textbf{Virtual Organization} (VO), in modo da poter \textbf{delegare} ad essa le operazioni da svolgere. 
    \\~\
    \item È disponibile un \textbf{modulo di NGINX} per l'\textbf{estrazioni dei dati} da un VOMS Proxy. 
  \end{itemize} 
  
\end{frame}

\section[Storm-Tape]{Interfacciamento con StoRM-Tape}
\begin{frame}
  \frametitle{StoRM-Tape}
\begin{itemize}
  \item Per dimostrare le qualità del PoC, si è deciso di \textbf{interfacciare} il sistema con un \textbf{componente middleware} attualmente in sviluppo al CNAF.  
  \\~\
  \item \textbf{StoRM-Tape} è un'implementazione della specifica della WLCG Tape REST API, che offre
  un'\textbf{interfaccia comune} per la gestione della \textbf{residenza dei dati} mantenuti negli \textbf{storage} dei grid computazionali.
\end{itemize}
\end{frame}

\begin{frame}
  \frametitle{Interfacciamento}
   Le \textbf{modifiche} necessarie per interfacciare il sistema con StoRM-Tape sono \textbf{minime}: \\~\
  \begin{itemize}
    \item \textbf{Modifica dei permessi} in modo che siano correlate con le funzionalità offerte da StoRM-Tape. \\~\
    \item \textbf{Ridenominazione} degli \textbf{endpoint} nel \textbf{reverse proxy} in modo che coincidano che quelle dell'API. 
  \end{itemize}
\end{frame}



\section[Conclusioni]{Conclusioni e sviluppi futuri}
\begin{frame}
  \frametitle{Conclusioni e sviluppi futuri}
  Il PoC realizzato \textbf{soddisfa tutti i requisiti} e si interfaccia con servizi di diverso carattere in modo \textbf{trasparente}, richiedendo \textbf{poche modifiche}. 
  \\ Alcune possibili evoluzioni:
\begin{itemize}
  \item Possibilità di \textbf{modificare le policy} collegandosi dall'\textbf{esterno} dell'infrastruttura.
  \item Modifica delle regole per permettere un controllo degli accessi \textbf{capability-based}.
  \item Variazione delle topologia in modo che il \textbf{servizio applicativo} possa comunicare \textbf{direttamente} con \textbf{OpenPolicyAgent}. 
\end{itemize}
\end{frame}

\end{document}